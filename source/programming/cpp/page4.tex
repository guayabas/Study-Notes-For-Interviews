%%%%%%%%%% Page 4 - Col 1 %%%%%%%%%%
\newpage
\colorfulheader{\texttt{C++} programming}

\begin{minipage}[t]{0.485\textwidth}
    \begin{itemize}
        \setlength\itemsep{1pt}
        \item \inlinecode{static} keyword in a class acts like a \quotes{global} variable
        that requires class scope for accessing
        \item A static member function of a class cannot access non-static members
        \begin{lstlisting}
            class Dummy 
            {
                int m; // Normal member
            public:
                static int n; // Requires definition in global space
                static void log_s() 
                { 
                    //printf("m : %d\n", m); Cannot access m member
                    //log(); Cannot access log function
                    printf("n : %d\n", Dummy::n);
                }
                void log_c() { printf("n, m : %d, %d\n", n, m); }
                Dummy(int _m) { m = _m;  n++; } ~Dummy() { n -= 2; }
            };
            int Dummy::n = 0;
            Dummy::log_s(); // Prints 0
            Dummy dummy1(0); // n increments 1
            dummy1.log_c(); // Prints m=0, n=1
            Dummy::log_s(); // Prints 1
            { 
                Dummy dummy2(1); // n increments 1
                dummy2.log_c(); // Prints m=1, n=2 
                Dummy::log_s(); // Prints 2
            } // Finish scope, dummy2 calls destructor
            dummy1.log_c(); // Prints m=0, n=0
            Dummy::log_s(); // Prints 0
        \end{lstlisting}
        \item \inlinecode{const} blocks the availability to change values, either by instantiation or by getting values
        \begin{lstlisting}
            class MyClass 
            {
            public:
                int x;
                MyClass(int _x) : x(_x) {};
                int get_nonconst() { return x; }
                // x++; Not allow to modify content in function
                int get_const1() const { return x; }
                // Allow to modify, but returns a const number
                const int get_const2() 
                { 
                    x++; const int y = x; 
                    return y; 
                }
            };
            const MyClass MyFoo(10);
            MyClass MyBar(20);
            // MyFoo.x = 20; Not valid, MyFoo is const
            // MyFoo.get_nonconst(); Not valid, non const method
            // MyFoo.get_const2(); Also not valid
            int x = MyFoo.get_const1(); // 10
            int y = MyBar.get_nonconst(); // 20
            int z = MyBar.get_const1(); // 20
            int w = MyBar.get_const2(); // 21
        \end{lstlisting}
        Possible options are
        \begin{lstlisting}
            // const member function
            int get() const { return x; }
            // member function returning const &
            const int & get() { return x; }
            // const member function returning const &
            const int & get() const { return x; }
        \end{lstlisting}
    \end{itemize}
\end{minipage}
%%%%%%%%%%%%%%%%%%%%%%%%%%%%%%%%%%%%
%%%%%%%%%% Page 4 - Col 2 %%%%%%%%%%
\begin{minipage}[t]{0.485\textwidth}
    \begin{itemize}
        \setlength\itemsep{0pt}
        \item Suppose \inlinecode{C} a class
        \begin{customcenter}[0pt]
            \begin{tabular}{|l|l|}
                \hline
                \textbf{Class special member} & \textbf{Syntax} \\
                \hline
                Default Constructor & \inlinecode{C::C\parenthesis{};} \\
                \hline
                Destructor & \inlinecode{C::$\sim$C\parenthesis{};} \\
                \hline
                Copy Constructor & \inlinecode{C::C\parenthesis{const C\&};} \\
                \hline
                Copy Assignment & \inlinecode{C\& operator=\parenthesis{const C\&};} \\
                \hline
                Move Constructor & \inlinecode{C::C\parenthesis{C\&\&};} \\
                \hline
                Move Assignment & \inlinecode{C\& operator=\parenthesis{C\&\&};} \\
                \hline
            \end{tabular}
        \end{customcenter}
        \begin{lstlisting}
            class C {
            private: int x = 0;
            public:
                C() { printf("Constructor C\n"); } 
                ~C() { printf("Destructor C\n"); }
                C(const C&) { printf("Copy Constructor C\n"); }
                C& operator=(const C&) 
                { printf("Copy Assignment C\n"); }
                C(C&&) { printf("Move Constructor C\n"); }
                C& operator=(C&&) { printf("Move Assignment"); }
                void log(char s) 
                { printf("%c : %p -> %d(%p)\n", s, this, x, &x); }
            };
            printf("%d\n", sizeof(C));
            C objectC_a = C();
            objectC_a.log('A');
            C objectC_b = C(objectC_a);
            objectC_b.log('B');
            objectC_a.log('A');
            C objectC_c = C(std::move(objectC_a));
            objectC_c.log('C');
            objectC_b.log('B');
            objectC_a.log('A'); 
        \end{lstlisting}
    \end{itemize}
    \textbf{Class Templates}
    \begin{itemize}
        \setlength\itemsep{0pt}
        \item Declaration $\mapsto$ \inlinecode{template <class T> class class\_name\curlybracket{\hspace{4pt}};} \\ Instantiation $\mapsto$ \inlinecode{class\_name<T> object\_name\parenthesis{\hspace{4pt}};}
        \begin{lstlisting}
            template <class T> class Pair {
                T values[2];
            public:
                Pair(T a, T b) { values[0] = a; values[1] = b; }
                void log() {
                    printf("%d bytes\n", sizeof(T));
                    printf("[0] %p->%f\n", &values[0], values[0]);
                    printf("[1] %p->%f\n", &values[1], values[1]); 
                }
            };
            // Creates a Pair float, Pair double, respectively
            Pair<float> pairFloat(1.0f, 2.0f); pairFloat.log();
            Pair<double> pairDouble(3.0, 4.0); pairDouble.log();
        \end{lstlisting}
        \item Can be specialized for specific type of data types
        \begin{lstlisting}
            template <typename T> class Foo {
                public: Foo(T arg) 
                { printf("Generic template : %s\n", arg.c_str()); } 
            };
            template <> class Foo<char> {
                public: Foo(char arg) 
                { printf("Specialized template : %c\n", arg); } 
            };
            Foo<std::string>(std::string("hello std::string"));
            Foo<char>('A');
        \end{lstlisting}
    \end{itemize}
\end{minipage}
%%%%%%%%%%%%%%%%%%%%%%%%%%%%%%%%%%%%