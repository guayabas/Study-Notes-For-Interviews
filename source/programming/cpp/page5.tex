%%%%%%%%%% Page 5 - Col 1 %%%%%%%%%%
\newpage
\colorfulheader{\texttt{C++} programming}

\begin{minipage}[t]{0.500\textwidth}
    \colorfulsection{Inheritance}
    \begin{itemize}[leftmargin=*]
        \setlength\itemsep{0pt}
        \item \inlinecode{class derived\_class :}\\ \inlinecode{\optionalkeyword{qualifier} base\_class\_name \curlybracket{\hspace{4pt}}; }
        \item Base class can be an \textbf{interface} using the \inlinecode{virtual} keyword and the \inlinecode{= 0} post-declaration to get \textbf{pure virtual} behavior
        \item The \inlinecode{\optionalkeyword{qualifier}} can be \inlinecode{public}, \inlinecode{protected}, or \inlinecode{private}
        \begin{center}
            \begin{tabular}{|l|c|c|c|}
                \hline
                \textbf{Access} & \inlinecode{public} & \inlinecode{protected} & \inlinecode{private} \\
                \hline
                base members & Y & Y & Y\\
                \hline
                derived members & Y & Y & N\\
                \hline
                not members & Y & N & N \\
                \hline
            \end{tabular}
        \end{center}
        \begin{lstlisting}
        namespace MX
        {
            class Polygon 
            {
            protected:
                int w = 0, h = 0;
            public: 
                Polygon() {}
                ~Polygon() { printf("Destructor Polygon\n"); }
                Polygon(int w, int h) : w(w), h(h) {}                
                virtual float area() = 0; // pure virtual function
                void log() { printf("Area : %f\n", this->area()); }
            };
            class Rectangle : public Polygon {
            public:
                Rectangle(int w = 1, int h = 1) : Polygon(w, h) {}
                ~Rectangle() { printf("Destructor Rectangle\n"); }
                float area() { return float(w * h); }
            };
            class Triangle : public Polygon {
            public:
                Triangle(int w = 1, int h = 1) : Polygon(w, h) {}
                float area() { return (w * h * 0.5f); }
            };
            class Pentagon : public Polygon {
            public:
                Pentagon(int side = 1) {
                    // TODO: Compute w,h for a pentagon using side
                    this->w = 0; this->h = 0; }
            };
        }
        // auto polygon = MX::Polygon(0, 0); Error, Pure virtual
        // Polymorphism (* base_class = * derived_class)
        MX::Polygon * rectangle1 = new MX::Rectangle();
        rectangle1->log();
        delete rectangle1; // Call to MX::~Polygon()
        { auto rectangle2 = MX::Rectangle(2, 2); 
        // Call to MX::~Rectangle and then MX::~Polygon()
        rectangle2.log(); }
        auto triangle = MX::Triangle();
        // Call to MX::~Polygon, MX::~Triangle() not defined
        triangle.log();
        // MX::Pentagon(); Error, area() does not have overrider
        \end{lstlisting}
    \end{itemize}
\end{minipage}
%%%%%%%%%%%%%%%%%%%%%%%%%%%%%%%%%%%%
\hspace{10pt}
%%%%%%%%%% Page 5 - Col 2 %%%%%%%%%%
\begin{minipage}[t]{0.470\textwidth}
    \colorfulsection{STL (Standard Template Library) Containers}
    \begin{itemize}[leftmargin=*]
        \setlength\itemsep{0pt}
        \item A \textbf{container} is an object that \textbf{stores a collection} of other objects
        \item The container \textbf{manages the storage space} for its elements and gives member \textbf{functions to access} them
        \item STL uses the namespace \inlinecode{std}
        \begin{customcenter}[0pt]
            \begin{tabular}{|l|l|}
                \hline
                \multicolumn{2}{|c|}{\textbf{Sequence Containers}} \\
                \multicolumn{2}{|c|}{Data structures that can be accessed sequentially} \\
                \hline
                \cellcolor{blue!15}\inlinecode{array} & Static contiguous array \\
                \cellcolor{blue!15}\inlinecode{vector} & Dynamic contiguous array \\
                \cellcolor{blue!15}\inlinecode{deque} & Double-ended queue \\
                \cellcolor{blue!15}\inlinecode{forward\_list} & Singly-linked list \\
                \cellcolor{blue!15}\inlinecode{list} & Doubly-linked list \\
                \hline
                \multicolumn{2}{|c|}{\textbf{Average search} requires $O\parenthesisA{n}$} \\
                \hline
            \end{tabular}
        \end{customcenter}
        \begin{customcenter}[0pt]
            \begin{tabular}{|l|l|}
                \hline
                \multicolumn{2}{|c|}{\textbf{\optionalkeyword{Ordered} Associative Containers}} \\
                \multicolumn{2}{|c|}{Data structure that is kept sorted (by keys)} \\
                \hline
                \cellcolor{blue!15}\inlinecode{set} & Collection of unique keys \\
                \cellcolor{blue!15}\inlinecode{map} & Collection of key-value pairs \\[-2pt]
                \cellcolor{blue!15} & with unique keys \\
                \cellcolor{blue!15}\inlinecode{multiset} & Collection of keys \\
                \cellcolor{blue!15}\inlinecode{multimap} & Collection of key-value pairs \\
                \hline
                \multicolumn{2}{|c|}{\textbf{Average search} requires $O\parenthesisA{\log n}$} \\
                \hline
            \end{tabular}
        \end{customcenter}
        \begin{customcenter}[0pt]
            \begin{tabular}{|l|l|}
                \hline
                \multicolumn{2}{|c|}{\textbf{Unordered Associative Containers}} \\
                \multicolumn{2}{|c|}{Unsorted (hashed by keys) data structure} \\
                \hline
                \cellcolor{blue!15}\inlinecode{unordered\_set} & Collection of unique keys \\
                \cellcolor{blue!15}\inlinecode{unordered\_map} & Collection of key-value pairs \\
                \cellcolor{blue!15} &  with unique keys \\
                \cellcolor{blue!15}\inlinecode{unordered\_multiset} & Collection of keys \\
                \cellcolor{blue!15}\inlinecode{unordered\_multimap} & Collection of key-value pairs \\
                \hline
                \multicolumn{2}{|c|}{\textbf{Average search} requires $O\parenthesisA{1}$ amortized} \\
                \multicolumn{2}{|c|}{\textbf{Worst case search} requires $O\parenthesisA{n}$ amortized} \\
                \hline
            \end{tabular}
        \end{customcenter}
        \begin{customcenter}[0pt]
            \begin{tabular}{|l|l|}
                \hline
                \multicolumn{2}{|c|}{\textbf{Container adaptors}} \\
                \multicolumn{2}{|c|}{Different interface for sequential containers} \\
                \hline
                \cellcolor{blue!15}\inlinecode{stack} & LIFO data structure \\
                \cellcolor{blue!15}\inlinecode{queue} & FIFO data structure \\
                \cellcolor{blue!15}\inlinecode{priority\_queue} & First element is the greatest and  \\[-2pt]
                \cellcolor{blue!15}& elements are in nonincreasing order \\
                \hline
                \multicolumn{2}{|c|}{\textbf{Average search} requires $O\parenthesisA{n}$} \\
                \hline
            \end{tabular}
        \end{customcenter}
        \begin{customcenter}[0pt]
            \begin{tabular}{|l|l|l|}
                \hline
                & \cellcolor{blue!15}\inlinecode{vector} & \cellcolor{blue!15}\inlinecode{\optionalkeyword{forward\_}list} \\
                \hline
                \textbf{Memory} & Contiguous block & Random order \\
                \hline
                \textbf{Access} & Constant & Linear \\
                \hline
                \textbf{Element} & \multirow{2}{*}{None} & Pointers to \\
                \textbf{Overhead} & & \texttt{next} and \optionalkeyword{\texttt{prev}} \\
                \hline
                \textbf{Resize} & Linear & Constant \\
                \hline
                \textbf{Safety} & Thread safe & Not thread safe \\
                \hline
                \textbf{Insertion} & $\mathcal{O}\parenthesisA{1}$ at end\textcolor{white}{aaa} & \multirow{2}{*}{At most $\mathcal{O}\parenthesisA{n}$} \\
                \textbf{Deletion} & $\mathcal{O}\parenthesisA{n}$ otherwise & \\
                \hline
            \end{tabular}
        \end{customcenter}
        \begin{customcenter}[0pt]
            \begin{tabular}{|l|l|l|}
                \hline
                & \cellcolor{blue!15}\inlinecode{unordered\_map} & \cellcolor{blue!15}\inlinecode{map} \\
                \hline
                Order & Non-Sorted & Sorted \\
                \hline
                Implementation & Hash Table & Red-Black Tree \\
                \hline
                \multirow{2}{*}{Search} & $\mathcal{O}\parenthesisA{1}$ average & \multirow{2}{*}{$\mathcal{O}\parenthesisA{\log n}$} \\
                & $\mathcal{O}\parenthesisA{n}$ worst & \\
                \hline
                Insertion & \multirow{2}{*}{Same as search} & $\mathcal{O}\parenthesisA{\log n}$ \\
                Deletion & & + Rebalancing \\
                \hline
            \end{tabular}
        \end{customcenter}
    \end{itemize}
\end{minipage}
\newpage
\begin{minipage}[t]{0.485\textwidth}
    \colorfulsection{Miscellaneous}
    \begin{itemize}[leftmargin=*]
        \setlength\itemsep{0pt}
        \item \textbf{Variables} are by \textbf{default mutable}, use the \inlinecode{const} keyword to make them immutable
        \item The keyword \inlinecode{auto} can be useful to infer data types
        \begin{lstlisting}
            #include <ComplicatedLibraryAPI.h>
            // Don't know type returning by API but can be deduced
            auto MethodInAPI = IntersectionObjectAPI();
            auto x = 1; // Infers x as an int
            vector<int> v{1,2,3};
            // STL iterator type is verbose, deduce it with auto
            auto bi = v.begin()
        \end{lstlisting}
        \item \inlinecode{constexpr} means \quotes{\textbf{evaluate at compile time}}
        \item RAII stands for \textbf{Resource Allocation It is Initialized}. It is an \quotes{abstract concept} for allocation$|$destruction of objects. An example are \inlinecode{unique\_ptr} and \inlinecode{shared\_ptr}
        \item \textbf{Move semantic} is done by providing \textbf{Move constructor} and \textbf{Move assignment}
        \item A move semantic allows \textbf{shallow copy}
        \item The \inlinecode{std::move()} does move semantic but treat it as a \quotes{cast} and once an object is moved is considered expired
        \item \inlinecode{variant} is a type-safe union
        \item Prefer \textbf{separate compilation}
        \begin{customcenter}[0pt]
            \begin{tabular}{|c|c|c|}
                \hline
                \texttt{Usage.cpp} & \texttt{Vector.cpp} & \texttt{Vector.h}\\
                \hline
                Instatiation and use of & Implementation & \multirow{2}{*}{Interface} \\
                objects from \texttt{Vector.cpp} & details & \\
                \hline
            \end{tabular}
        \end{customcenter}
        \item \textbf{Avoid macros}. Including same header in multiple files can affect the meaning$|$behavior of a macro
        \item STL algorithms (such as find, copy, sort, etc.) have a \textbf{parallel version} using the argument \inlinecode{std::execution::par}
        \begin{lstlisting}
            // Let v be an STL container
            sort(v.begin(), v.end()); // Default, sequential
            sort(std::execution::seq, v.begin(), v.end()); // Same as before
            sort(std::execution::par, v.begin(), v.end()); // Parallel version
        \end{lstlisting}
        \item \inlinecode{std::array} has same performance as a \texttt{C} style array
        \item For \inlinecode{vector}, the \textbf{access operator \inlinecode{[]} does not check bounds} as opposed to \inlinecode{.at()} 
        \begin{lstlisting}
            vector<int> v{1,2,3,4};
            cout << v[10]; // Ok, will give some value but not throw
            cout << v.at(10); // Error, terminate with out_of_range 
        \end{lstlisting}
        \item \inlinecode{.data()} method, for STL containers that support it, acts like a pointer to the first element
        \begin{lstlisting}
            void usePointer(int * ptr) { ... };
            array<int,5> v{1,2,3,4,5};
            usePointer(v); // Error, v is not an int *
            usePointer(&v[0]); // Ok, pointer to 1st element
            usePointer(v.data()); // Similar as before
        \end{lstlisting}
        \item \inlinecode{std::valarray} is a vector-like container that supports \textbf{element-wise operations}, \textbf{slicing}, and \textbf{shifting}. These operations are similar to what \texttt{Python} containers can offer.
        \item Try \inlinecode{accumulate()}, \inlinecode{inner\_product()}, \inlinecode{partial\_sum()}, and \inlinecode{adjacent\_difference()} before writing a loop for computing a value from a sequence
    \end{itemize}
\end{minipage}
\hspace{10pt}
\begin{minipage}[t]{0.485\textwidth}
    \colorfulsection{Concurrency}
    \begin{itemize}[leftmargin=*]
        \setlength\itemsep{0pt}
        \item The execution of \textit{several tasks} is named \textbf{concurrency}. Used to improve
        \begin{enumerate}[leftmargin=*]
            \setlength\itemsep{0pt}
            \item \textbf{Throughput} $\mapsto$ Using several processors for a single computation
            \item \textbf{Responsiveness} $\mapsto$ One part of a program progresses while another waits for a response        
        \end{enumerate}
        \vspace{-3pt}
        \item \texttt{C++} \textbf{supports it via the STL} and is aimed to provide \textit{system$-$level concurrency} using multiple threads in a single address space (with a suitable \textit{memory model} and \textit{atomic operations})
        \item  A \textbf{task} is a computation that can \textbf{potentially be executed concurrently} with other computations. A \textbf{thread}, \inlinecode{std::thread}, is a system$-$level representation of a task in a program. A task is an argument for a thread
        \begin{customcenter}[0pt]
            \begin{lstlisting}
                void task1();
                struct task2
                {
                    void operator()();
                };
                thread t1{task1}; // task1() executed in thread i
                thread t2{task2()}; // task2()() executed in thread j
                t1.join() // wait for t1 to finish
                t2.join() // wait for t2 to finish
            \end{lstlisting}
        \end{customcenter}
        \item \inlinecode{.join()} ensures thread completion
        \item Given the single address space of threads, data can be communicated between them via \textbf{shared objects}. The communication is controlled by \textbf{locks}. A \textbf{process} differs from threads in not having (generally) shared data
        \item To pass (and$|$or) return variables one can do passing arguments or pointers
        \begin{customcenter}[0pt]
            \begin{lstlisting}
                typedef const vector<float> cvfloat;
                void f(cvfloat & v, float * res); // place result in res
                class F
                {
                public:
                    F(cvfloat & v, float * res) : v{v}, res{res} {}
                    void operator()(); // to be callable in thread input
                private:
                    cvfloat & v;
                    float * res;
                };
                float g(cvfloat & v); // use return value
                void Threads(cvfloat & v1, cvfloat & v2, cvfloat & v3)
                {
                    float res1, res2, res3;
                    thread t1{f, cref{v1}, &res1};
                    thread t2{F{v2, &res2}};
                    thread t3{[&](){res3 = g(v3)}; // capture reference
                    t1.join(); t2.join(); t3.join(); 
                }
            \end{lstlisting}
        \end{customcenter}
        \item The \inlinecode{cref()} function (or \inlinecode{ref()}) is a \textit{type function} from \inlinecode{<functional>} header required to pass a reference in the constructor of \texttt{thread}
        \item The syntax \inlinecode{[\&]()\{...\}} for \texttt{thread t3} is a \textbf{lambda expression}. In general form it can be defined as \\ \inlinecode{\bracket{capture clause}\parenthesis{parameters}$\mapsto$return type\curlybracket{definition}}
    \end{itemize}
\end{minipage}
\newpage
\begin{minipage}[t]{0.485\textwidth}
    \begin{itemize}[leftmargin=*]
        \setlength\itemsep{0pt}
        \item For sharing data between tasks use \inlinecode{mutex} (\textbf{mutual exclusive object}). A \texttt{thread} acquires a \texttt{mutex} via \inlinecode{lock()} and releases via \inlinecode{unlock()}
        \begin{customcenter}[0pt]
            \begin{lstlisting}
                mutex m;
                int sharedData;
                // RAII is being used in f() scope
                void f()
                {
                    scoped_lock threadLock{m}; // acquire mutex implicit
                    sharedData += 42;
                    // release mutex implicit
                }
            \end{lstlisting}
        \end{customcenter}
        \item Simultaneous access to resources can lead to \textbf{deadlocks}
        \begin{customcenter}[0pt]
            \begin{tikzpicture}
                \tikzstyle{arrow}=[thick,->,>=stealth]
                \node[draw=red,thick,minimum width=3.0cm,minimum height=3.0cm](n){};
                \node[below of=n,yshift=-0.75cm]{\textcolor{red}{Deadlock}};
                \node[draw,xshift=-3cm](n1){Thread 1};
                \node[draw,above of=n1,xshift=+3cm,minimum width=2.5cm,text centered](n11){Mutex 1};
                \node[draw,xshift=+3cm](n2){Thread 2};
                \node[draw,below of=n2,xshift=-3cm,minimum width=2.5cm,text centered](n21){Mutex 2};
                \draw[arrow,blue] (n1) |- node[anchor=south] {Acquire} (n11);
                \draw[arrow,orange] (n2) |- node[anchor=north] {Acquire} (n21);
                \draw[arrow,blue!70] ($(n11)-(1.0,0.25)$) -- node[anchor=south,rotate=-90] {Release} ($(n21)-(1.0,-0.25)$);
                \draw[arrow,blue!40] ($(n11)-(0.5,0.25)$) -- node[anchor=south,rotate=-90] {Acquire} ($(n21)-(0.5,-0.25)$);
                \draw[arrow,orange!70] ($(n21)+(1.0,0.25)$) -- node[anchor=south,rotate=+90] {Release} ($(n11)+(1.0,-0.25)$);
                \draw[arrow,orange!40] ($(n21)+(0.5,0.25)$) -- node[anchor=south,rotate=+90] {Acquire} ($(n11)+(0.5,-0.25)$);
            \end{tikzpicture}
        \end{customcenter}
        \item Besides \inlinecode{scoped\_lock} there are \inlinecode{unique\_lock} (unique access) and \inlinecode{shared\_lock} (able to share among threads)
    \end{itemize}
    \colorfulsection{\texttt{C++} versions features}
    \begin{itemize}[leftmargin=*]
        \setlength\itemsep{0pt}
        \item \texttt{C++11} language features\vspace{-4pt}
        \begin{itemize}[leftmargin=*,label=$\blacksquare$]
            \setlength\itemsep{0pt}
            \item Uniform and general initialization via \inlinecode{\curlybracket{}}\vspace{-2pt}
            \item Type deduction \inlinecode{auto}
            \item Guaranted constant expression \inlinecode{constexpr}
            \item Range \texttt{for-statement}
            \item \inlinecode{nullptr} as null pointer keyword\vspace{-2pt}
            \item Scoped types \inlinecode{enum} $\mapsto$ \inlinecode{enum class}
            \item Enabling move semantics
            \item Lambdas
            \item Variadic templates?
            \item Unicode characters
            \item \inlinecode{long long} integer type
            \item Memory alignment controls \inlinecode{alignas} and \inlinecode{alignof}
            \item Raw string literals
            \item \inlinecode{\_\_func\_\_} as name (string) of current function
            \item Namespaces \inlinecode{inline}?
            \item Inheriting constructors            
        \end{itemize}
        \item \texttt{C++14} language features\vspace{-4pt}
        \begin{itemize}[leftmargin=*,label=$\blacksquare$]
            \setlength\itemsep{0pt}
            \item Function return-type deductions
            \item Binary literals?
            \item Generic lambdas
            \item \inlinecode{[[deprecated]]} attribute
        \end{itemize}
        \item \texttt{C++17} language features\vspace{-4pt}
        \begin{itemize}[leftmargin=*,label=$\blacksquare$]
            \setlength\itemsep{0pt}
            \item Compile-time \inlinecode{if}
            \item \inlinecode{constexpr} lambdas\vspace{-2pt}
            \item Variables with \inlinecode{inline}?
            \item \inlinecode{[[fallthrough]]}, \inlinecode{[[nodiscard]]}, \inlinecode{[[maybe\_unused]]}
            \item \inlinecode{std::byte} type
        \end{itemize}
    \end{itemize}
\end{minipage}
\hspace{10pt}
\begin{minipage}[t]{0.485\textwidth}
    \begin{itemize}[leftmargin=*]
        \setlength\itemsep{0pt}
        \item \texttt{C++11} STL features\vspace{-4pt}
        \begin{itemize}[leftmargin=*,label=$\blacksquare$]
            \item Move semantics for containers
            \item Singly-linked list \inlinecode{forward\_list}
            \item Hash containers \\ \inlinecode{unordered\_map}, \inlinecode{unordered\_multimap} \\ \inlinecode{unordered\_set}, \inlinecode{unordered\_multiset}
            \item Smart pointer \inlinecode{unique\_ptr}, \inlinecode{shared\_ptr}, and \inlinecode{weak\_ptr}
            \item Concurrency \inlinecode{thread}, \inlinecode{mutex}, \texttt{locks}
            \item \inlinecode{tuple} container
            \item Regular expressions \inlinecode{regex}
            \item Random numbers \texttt{distributions} and \texttt{engines}
            \item Integer type names such as \inlinecode{int16\_t}, \inlinecode{uint32\_t}, etc.
            \item Fixed-size contiguous container \inlinecode{array}
            \item \inlinecode{string} to numeric values conversions
            \item Time utilities \inlinecode{duration} and \inlinecode{time\_point}
            \item Lower-level concurrency type \inlinecode{atomic}
        \end{itemize}
        \item \texttt{C++14} STL features\vspace{-4pt}
        \begin{itemize}[leftmargin=*,label=$\blacksquare$]
            \item \inlinecode{shared\_mutex}
            \item User-defined literals?
        \end{itemize}
        \item \texttt{C++17} STL features\vspace{-4pt}
        \begin{itemize}[leftmargin=*,label=$\blacksquare$]
            \item File system \texttt{<filesystem>}
            \item Parallel algorithms
            \item Mathematical special functions
            \item \inlinecode{variant} keyword
        \end{itemize}
    \end{itemize}
\end{minipage}
