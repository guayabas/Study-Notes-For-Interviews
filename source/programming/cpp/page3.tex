%%%%%%%%%% Page 3 - Col 1 %%%%%%%%%%
\newpage
\colorfulheader{\texttt{C++} programming}

\begin{minipage}[t]{0.485\textwidth}
    \colorfulsection{Data Structures}
    \begin{itemize}[leftmargin=*]
        \setlength\itemsep{0pt}
        \item \inlinecode{struct \optionalkeyword{type\_name}} \\
        \inlinecode{\curlybracket{ member\_type\_n member\_name\_n; } \optionalkeyword{object};}
        \item Accessing of members via the \inlinecode{.} operator
        \begin{lstlisting}
            struct data { int x; int y; int z; } coordinates;
            coordinates.x = 1; // Assigning 1 to member x
            // Assigning members y and z
            coordinates.y = coordinates.z = 0;
        \end{lstlisting}
        \item Use the \inlinecode{$\rightarrow$} to access the members of a pointer to struct. Alternatively, one can dereference and then access a member, i.e. \inlinecode{\parenthesis{*pointer\_to\_struct}.}
        \begin{lstlisting}
            struct fruits_t 
            { std::string apple; std::string banana; };
            fruits_t afruit;
            fruits_t * pfruit = &afruit;
            pfruit->apple = "apple"; // Access via -> operator
            // Access via dereference-member, i.e. (*). 
            (*pfruit).banana = "banana";
        \end{lstlisting}
        \item \inlinecode{class} keyword is similar to struct (in holding variables) but with different default access to members \\ \inlinecode{struct} $\mapsto$ default \textbf{public} access \\ \inlinecode{class} $\mapsto$ default \textbf{private} access
        \begin{lstlisting}
            class foo_ct { public: int x;  } foo_c;
            struct foo_st { int x; } foo_s;
            foo_c.x = 0; // Accessible with public: modifier
            foo_s.x = 0; // Accessible by default
        \end{lstlisting}
        \item Another keyword similar to struct is \inlinecode{union}. While struct (and class) allocates memory for each of its members, union will allocate only a chunk to be reused among members
        \begin{lstlisting}
        union mix_t { 
            int l; struct { short hi; short lo; }; char c[4]; 
        } mix_example;
        printf("%d\n", sizeof(mix_example)); // Prints 4
        \end{lstlisting}
    \end{itemize}
    \colorfulsection{Enumerated Types}
    \begin{itemize}[leftmargin=*]
        \setlength\itemsep{0pt}
        \item \inlinecode{enum \optionalkeyword{class} type\_name \optionalkeyword{:\hspace{4pt}data\_type}} \\ \inlinecode{\curlybracket{ value1 \optionalkeyword{=\hspace{4pt}init\_val}, value2, ... } \optionalkeyword{object\_names}; }
        \item Default underlying memory is \inlinecode{int}
        \begin{lstlisting}
            enum ColorEnumDefault { red = 42, green, blue };
            enum class ColorEnum : char { red, green, blue };
            printf("%d\n", red); // OK but easily to do name clash
            printf("%d\n", green); // Prints 43
            // Prints 4 and better intent for access
            printf("%d\n", sizeof(ColorEnumDefault::red));
            // Prints 1 and no ambiguity
            printf("%d\n", sizeof(ColorEnum::red));
            printf("%d\n", ColorEnum::green); // Prints 1
        \end{lstlisting}
    \end{itemize}
    \colorfulsection{Aliases}
    \begin{itemize}[leftmargin=*]
        \setlength\itemsep{0pt}
        \item Two keywords, \inlinecode{typedef} or \inlinecode{using} \\ \inlinecode{typedef existing\_type new\_type\_name;} \\ \inlinecode{using new\_type\_name = existing\_type;}
        \begin{lstlisting}
            typedef unsigned char uchar; using uint = unsigned int;
            uchar a = 0; // OK to use
            uint b = 1; // Also OK to use
        \end{lstlisting}
    \end{itemize}
\end{minipage}
%%%%%%%%%%%%%%%%%%%%%%%%%%%%%%%%%%%%
\hspace{10pt}
%%%%%%%%%% Page 3 - Col 2 %%%%%%%%%%
\begin{minipage}[t]{0.485\textwidth}
    \colorfulsection{Classes}
    \begin{itemize}[leftmargin=*]
        \setlength\itemsep{0pt}
        \item \inlinecode{class \optionalkeyword{class\_name}} \\
        \inlinecode{\curlybracket{ \optionalkeyword{access\_n :}\hspace{4pt}type\_m member\_name\_m; } \optionalkeyword{object};}
        \item Creation of object $\mapsto$ Constructor \inlinecode{\parenthesis{}, =, \curlybracket{}} \\ Destruction of object $\mapsto$ Destructor \inlinecode{$\sim$}
        \item Keyword \inlinecode{explicit} avoids implicit conversions in classes. Use it for constructor that takes one argument unless there is a good reason no to
        \begin{lstlisting}
            class Vector { public: explicit Vector(int size); };
            Vector v1(7); // OK, use defined constructor
            Vector v2 = 7 // Error, operator = not made by compiler
        \end{lstlisting}
        \item Keyword \inlinecode{this} is a pointer to the object whose member functions is being executed
        \begin{lstlisting}
            class Rectangle 
            {
                int w, h;
            public:
                void log_values() { printf("w:%d, h:%d\n", w, h); }
                void set(int, int);
                // This function could be inlined
                int area() { return w * h; }
            };
            void Rectangle::set(int _w, int _h) 
            {
                // w, h are private but accessible in class scope
                // *this* keyword represents members of the class
                this->w = _w;
                h = _h; 
            }
            // OK, call default constructor but might init garbage
            Rectangle r_1;
            r_1.log_values();
            // Warning, r_2() is calling a function not constructor
            Rectangle r_2();
            // Error, r_2 has not been created as Rectangle
            // r_2.log_values();
            // Also default constructor but init to zero
            Rectangle r_3{};
            r_3.log_values();
        \end{lstlisting}
        \item Class instantiation can be done in multiple forms : \textbf{default or functional}, \textbf{assignment}, \textbf{copy}. \textbf{move}, \textbf{list}
        \begin{lstlisting}
            class Circle 
            {
                double r;
            public:
                void log_values() { printf("%f\n", r); }
                Circle(double _r) { r = _r; }
                Circle& operator=(const Circle& c) 
                { r = c.r; return *this; } 
            };
            // Default construct is not available
            // due to defining one constructor with a parameter
            Circle foo(10.0); // Functional form
            Circle bar = 20.0; // Default assignment
            Circle but = bar; // Defined assignment?
            Circle baz{ 30.0 }; // Uniform init
            Circle qux = { 40.0 }; // ?
            foo.log_values(); // 10
            bar.log_values(); but.log_values(); // 20
            baz.log_values(); qux.log_values(); // 30, 40
        \end{lstlisting}
    \end{itemize}
\end{minipage}
%%%%%%%%%%%%%%%%%%%%%%%%%%%%%%%%%%%%