%%%%%%%%%% Page 1 - Col 1 %%%%%%%%%%
\newpage
\colorfulheader{\texttt{C++} programming}

\begin{minipage}[t]{0.485\textwidth}
    \colorfulsection{Data Types}
    \emptyline
    \begin{customcenter}[0pt]
        \begin{tabular}{l|c|c|l}
            & \multicolumn{2}{c|}{Bytes [\texttt{sizeof()}]} & \\
            \hline
            Keyword & \texttt{g++7.5} & \texttt{msvc++19} & Range \\
            \hline
            \texttt{bool} & 1 & 1 & [True,False]\\
            \texttt{unsigned char} & 1 & 1 & $[0, 2^8)$\\
            \texttt{\optionalkeyword{(signed)} char} & 1 & 1 & $[2^7 - 1, 2^7)$ \\
            \texttt{unsigned int} & 4 & 4 & $[0, 2^{32})$ \\
            \texttt{\optionalkeyword{(signed)} int} & 4 & 4 & $[2^{31} - 1, 2^{31})$ \\
            \texttt{long} & 8 & 4 & $[2^{63} - 1, 2^{63})$ \\
            \texttt{float} & 4 & 4 & Implementation-based \\
            \texttt{double} & 8 & 8 & Implementation-based \\
            \texttt{void} & \multicolumn{2}{c|}{Does not apply} & Does not apply \\
        \end{tabular}
    \end{customcenter}
    \colorfulsection{Operators}\vspace{-5pt}
    \begin{itemize}[leftmargin=*]
        \setlength\itemsep{0pt}
        \item Order of assignment is \textbf{left to right}, \inlinecode{int x = y = 42;}
        \item Arithmetic \inlinecode{+, -, *, /, \%}
        \item Bitwise \inlinecode{\&, |, $\wedge$, $\sim$, <<, >>}
        \item Logical \inlinecode{!, \&\&, ||}
        \item Relational \inlinecode{==, !=, >, <, >=, <=}
        \item $\substack{\text{Compound} \\ \text{Assignment}}$ \inlinecode{+=, -=, *=, /=, \%=} \inlinecode{<<=, >>=, \&=, |=}
        \item Pre-Post (In/De)crement
        {
            \begin{lstlisting}
                int x = 3;
                int y = ++x; // y contains 4
                int z = x++; // z contains 3
            \end{lstlisting}
        }
        \item Ternary \inlinecode{condition ?\hspace{5pt}option1:option2;}
        \item Object size in bytes \inlinecode{sizeof()}
    \end{itemize}
    \colorfulsection{Statement and Flow Control}
    \begin{itemize}[leftmargin=*]
        \setlength\itemsep{0pt}
        \item Generic statements \inlinecode{\curlybracket{statement1; statement2; statement3;}}
        \item Control if \inlinecode{if \parenthesis{condition} statement;}
        \item Control while \inlinecode{while \parenthesis{expression} statement;}
        \item Control do-while \inlinecode{do \curlybracket{statement} while \parenthesis{condition};}
        \item For iteration \inlinecode{for \parenthesis{declaration :\hspace{5pt}range} statement;}
        \item Jump statements \inlinecode{break; continue; goto
        \optionalkeyword{label};}
        \item Selection
        {
            \begin{lstlisting}
                switch {expression}
                {
                    case constant1: statement; break;
                    case constant2: statement; break;
                    ...
                    default: statement; break;
                }
            \end{lstlisting}
        }
    \end{itemize}
    \colorfulsection{Functions}
    \emptyline
    \inlinecode{type name\parenthesis{parameter1, parameter2, ...}\curlybracket{ statements; }}
    \begin{itemize}[leftmargin=*]
        \setlength\itemsep{0pt}
        \item Parameters are passed \textbf{by value} (creates a copy) \textbf{by default}
        \item \inlinecode{\optionalkeyword{const} type \&} to pass by reference (makes alias)
        \item \inlinecode{\optionalkeyword{const} type *} to pass pointer (memory location)
        \item Can be optimized using the \inlinecode{inline} keyword
    \end{itemize}
\end{minipage}
%%%%%%%%%%%%%%%%%%%%%%%%%%%%%%%%%%%%
\hspace{10pt}
%%%%%%%%%% Page 1 - Col 2 %%%%%%%%%%
\begin{minipage}[t]{0.485\textwidth}
    \begin{itemize}[leftmargin=*]
        \setlength\itemsep{0pt}
        \item It can be made \textbf{recursive}
        \item Same name but different parameters types define \textbf{overloading}
        \begin{lstlisting}
            void prototype(unsigned int x) {};
            void prototype(float x) {};
            int main() 
            {
                prototype(0); // Call unsigned int version
                prototype(0.0f); // Call float version
                return; 
            }
        \end{lstlisting}
        \item Deciding correct function to use between 2 (or more) objects at \textbf{runtime} is done via \textbf{virtual} keyword
        \item Compiler decides virtual functions via \textbf{vtable}
        \begin{customcenter}[-5pt]
            \begin{tikzpicture}
                \draw[step=1cm,white,very thin] (0,0) grid (9.5,2.75);
                \node[] (nodea) at (1.0, -0.1) {$\substack{\text{\textbf{Overhead of}}\\\text{\textbf{one pointer size}}}$};
                \node[rectangle, minimum width=2cm, minimum height=0.75cm, text centered, draw=black, fill=cyan!10] (nodev) at (1.25,0.6) {$\substack{\text{\textbf{V}}\\\text{Vector\_container}}$};
                \node[rectangle, minimum width=2cm, minimum height=0.75cm, text centered, draw=black, fill=cyan!10] (nodel) at (1.25,2.15) {$\substack{\text{\textbf{L}}\\\text{List\_container}}$};
                \begin{scope}
                    \node[right of=nodev,xshift=1.5cm,yshift=0.425cm] (nodev1) {\scriptsize \textbf{vtable}};
                    \node[rectangle,below of=nodev1,minimum width=1cm,minimum height=0.4cm,draw,yshift=0.6cm] (nodev2) {};
                    \node[rectangle,below of=nodev2,minimum width=1cm,minimum height=0.4cm,draw,yshift=0.6cm] (nodev3) {};
                    \node[rectangle,below of=nodev3,minimum width=1cm,minimum height=0.4cm,draw,yshift=0.6cm] (nodev4) {};
                    \node[rectangle,right of=nodev2,minimum height=0.1cm,draw,xshift=2.5cm] (nodev2t) {\tiny Vector\_container::operator[]()};
                    \node[rectangle,right of=nodev3,minimum height=0.1cm,draw,xshift=2.5cm] (nodev3t) {\tiny Vector\_container::size()};
                    \node[rectangle,right of=nodev4,minimum height=0.1cm,draw,xshift=2.5cm] (nodev4t) {\tiny Vector\_container::$\sim$Vector\_container()};
                \end{scope}
                \begin{scope}
                    \node[right of=nodel,xshift=1.5cm,yshift=0.4cm] (nodel1) {\scriptsize \textbf{vtable}};
                    \node[rectangle,below of=nodel1,minimum width=1cm,minimum height=0.4cm,draw,yshift=0.6cm] (nodel2) {};
                    \node[rectangle,below of=nodel2,minimum width=1cm,minimum height=0.4cm,draw,yshift=0.6cm] (nodel3) {};
                    \node[rectangle,below of=nodel3,minimum width=1cm,minimum height=0.4cm,draw,yshift=0.6cm] (nodel4) {};
                    \node[rectangle,right of=nodel2,minimum height=0.1cm,draw,xshift=2.5cm] (nodel2t) {\tiny List\_container::operator[]()};
                    \node[rectangle,right of=nodel3,minimum height=0.1cm,draw,xshift=2.5cm] (nodel3t) {\tiny List\_container::size()};
                    \node[rectangle,right of=nodel4,minimum height=0.1cm,draw,xshift=2.5cm] (nodel4t) {\tiny List\_container::$\sim$List\_container()};
                \end{scope}
                \draw[arrow] (nodev) -- (nodev2);
                \draw[arrow] (nodel) -- (nodel2);
                \draw[arrow] (nodev2) -- (nodev2t);
                \draw[arrow] (nodev3) -- (nodev3t);
                \draw[arrow] (nodev4) -- (nodev4t);
                \draw[arrow] (nodel2) -- (nodel2t);
                \draw[arrow] (nodel3) -- (nodel3t);
                \draw[arrow] (nodel4) -- (nodel4t);
                \draw[arrow] (nodea.west) |- (nodev);
                \draw[arrow] (nodea.west) |- (nodel);
            \end{tikzpicture}
        \end{customcenter}
        \item \textbf{Templates} maintain a method independent of the data type \\ \inlinecode{template <typename T>} \\ \inlinecode{T f(\optionalkeyword{const} T a, \optionalkeyword{const} T b) \curlybracket{ statements; } }
        \item To call a template function \inlinecode{f \optionalkeyword{<template\_args>}\parenthesis{f\_args};} 
        \begin{lstlisting}
            // The following template is to avoid writing
            // 1. int sum(int a, int b) { return a + b; }
            // 2. float sum(float a, float b) { return a + b; }
            template <typename T> T sum(T a, T b) { return a + b; }
            int main() 
            {
                sum(1, 2) // Deduces to use int version of template
                sum(1.0f, 2) // Will not compile due miss inferencing
                sum<float>(1.0f, 2.0f) // Redundant infer but OK 
                return; 
            }
        \end{lstlisting}
        \item Templates must be determined at \textbf{compile time}
        \begin{lstlisting}
            template <class T, int N>
            T multiply(T value) { return value * N; }
            int main 
            {
                multiply<int, 2>(10); // At compile time returns 20
                multiply<int, 3>(10); // Similar but returning 30
                return; 
            }
        \end{lstlisting}
    \end{itemize}
    \colorfulsection{Scopes}\vspace{-5pt}
    \begin{itemize}[leftmargin=*]
        \setlength\itemsep{0pt}
        \item \textbf{static} storage $\mapsto$ global variable $\mapsto$ init to zero
        \item \textbf{automatic} storage $\mapsto$ local variable $\mapsto$ undetermined init
        \item outside any block $\mapsto$ global scope $\mapsto$ \textbf{global variable}
        \item within a block $\mapsto$ block scope $\mapsto$ \textbf{local variable}
        \begin{lstlisting}
        int foo; // global variable
        int do_something() { int foo; } // local variable 
        \end{lstlisting}
        \item Name collisions can be prevented with namespace keyword \inlinecode{namespace identifier \curlybracket{ named\_objects }}
        \item Access of namespace is with qualifier \inlinecode{::} \\ \inlinecode{namespace gg \curlybracket{ int x }} $\mapsto$ \inlinecode{gg::x} access x
     \end{itemize}
\end{minipage}
%%%%%%%%%%%%%%%%%%%%%%%%%%%%%%%%%%%%