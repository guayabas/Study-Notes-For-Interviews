%%%%%%%%%% Page 2 - Col 1 %%%%%%%%%%
\newpage
\colorfulheader{computer graphics}

\begin{minipage}[t]{\textwidth}
    \vspace{-15pt}
    \begin{customcenter}[0pt]
        \colorfulsection{OpenGL Core Pipeline}
        \begin{tikzpicture}
            \tikzstyle{arrow}=[thick,->,>=stealth]
            % \draw[step=1cm,blue,very thin] (0,0) grid (20,+3);
            % \draw[step=1cm,red ,very thin] (0,0) grid (20,-3);
            \begin{scope}[xshift=0.8cm]
                \node (descr1symbol) [draw,ellipse,yshift=-2.5cm,fill=blue!30] {\textcolor{blue!30}{a}};
                \node (descr1text) [right of=descr1symbol,xshift=0.95cm] {Supplied by the user};
                \node (descr2symbol) [draw,ellipse,yshift=-2.5cm,xshift=4.5cm,fill=green!30] {\textcolor{green!30}{a}};
                \node (descr2text) [right of=descr2symbol,xshift=0.75cm] {Mandatory stage};
                \node (descr3symbol) [draw,ellipse,yshift=-2.5cm,xshift=8.5cm,fill=red!30] {\textcolor{red!30}{a}};
                \node (descr3text) [right of=descr3symbol,xshift=0.85cm] {Non-programmable};
                % Start of Pipeline
                \node (input) [draw,ellipse,minimum height=1cm,text centered,fill=blue!30] {\footnotesize{VAO}};
                \node (gi) [draw,rounded corners,above of=input,fill=cyan!30,minimum height=1cm,yshift=0.25cm] {$\substack{\text{Geometry}\\\text{Information}}$};
                \node (input-text) [above of=gi,yshift=-0.25cm] {\textbf{INPUT}};
                \node (vp) [draw,rounded corners,minimum width=2.7cm,minimum height=1cm,right of=input,text centered,fill=green!30,xshift=1.35cm] {$\substack{\text{Vertex}\\\text{Processor}}$};
                \node [draw,rounded corners,minimum width=2.4cm,minimum height=0.1cm,below of=vp,yshift=0.2725cm,fill=blue!30] {\scriptsize{Vertex Specification}};

                \node (vs) [draw,ellipse,minimum height=1cm,minimum width=2cm,above of=vp,fill=blue!30,yshift=0.25cm] {$\substack{\text{Vertex}\\\text{Shader}}$};

                \node (vpp)[draw,minimum width=2.0cm,minimum height=1cm,fill=red!30,right of=vp,xshift=1.75cm] {$\substack{\text{Vertex}\\\text{Postprocessor}}$};
                \node (pa) [draw,minimum width=2cm,minimum height=1cm,right of=vpp,text centered,fill=red!30,xshift=1.4cm] {$\substack{\text{Primitive}\\\text{Assembly}}$};

                \node (gs) [draw,ellipse,minimum height=1cm,minimum width=2cm,above left of=pa,yshift=0.5cm,fill=blue!30,xshift=-0.5cm] {$\substack{\text{Geometry}\\\text{Shader}}$};
                \node (ts) [draw,ellipse,minimum height=1cm,minimum width=2cm,above right of=pa,yshift=0.5cm,fill=blue!30,xshift=+0.5cm] {$\substack{\text{Tessellation}\\\text{Shader}}$};

                \node (ra) [draw,minimum width=1.65cm,minimum height=1cm,right of=pa,text centered,fill=red!30,xshift=1.25cm] {\footnotesize{Rasterizer}};
                \node (fp) [draw,rounded corners,minimum width=2.7cm,minimum height=1cm,right of=ra,text centered,fill=blue!30,xshift=1.6cm] {$\substack{\text{Fragment}\\\text{Processor}}$};
                \node (fs) [draw,ellipse,minimum height=1cm,minimum width=2cm,above of=fp,fill=blue!30,yshift=0.25cm,xshift=-1cm] {$\substack{\text{Fragment}\\\text{Shader}}$};
                \node (td) [draw,ellipse,minimum height=1cm,minimum width=2cm,above of=fp,fill=blue!30,yshift=0.25cm,xshift=1cm] {$\substack{\text{Texture}\\\text{Data}}$};
                \node (fb) [draw,minimum width=2.5cm,minimum height=1cm,right of=fp,text centered,fill=blue!30,xshift=2.0cm] {\footnotesize{Framebuffer}};
                \node (fbo) [draw,ellipse,minimum width=1cm,minimum height=1cm,fill=blue!30,above of=fb,yshift=0.25cm] {\footnotesize{FBO}};            
                \node (dt) [draw,fill=blue!30,below of=fb,minimum height=1cm,minimum width=1.0cm,yshift=-0.25cm,xshift=-1.4cm] {$\substack{\text{Depth}\\\text{Test}}$};
                \node (ab) [draw,fill=blue!30,below of=fb,minimum height=1cm,minimum width=1.5cm,yshift=-0.25cm,xshift=+0.0cm] {$\substack{\text{Alpha}\\\text{Blending}}$};
                \node (so) [draw,fill=blue!30,below of=fb,minimum height=1cm,minimum width=1.5cm,yshift=-0.25cm,xshift=+1.7cm] {$\substack{\text{Stencil}\\\text{Operations}}$};
                \draw [decorate,decoration = {brace,mirror,amplitude=10pt},ultra thick] (13.325,-1.8) -- (18.0,-1.8) node[black,pos=0.5,below=8pt,anchor=north]{$\substack{\text{Fragment}\\\text{Sample}}$ Operations (Seen, Discard, Scissor, etc.)};
                \node (display) [rounded corners,draw,minimum width=2.0cm,minimum height=1cm,fill=cyan!30,right of=fb,xshift=1.7cm] {\footnotesize{Display}};
                \node (output-text) [above of=display,yshift=-0.25cm] {\textbf{OUTPUT}};
                % End of pipeline
                % connections
                \draw[arrow] (input) -- (vp);
                \draw[arrow] (vp) -- (vpp);
                \draw[arrow] (vpp) -- (pa);
                \draw[arrow] (pa) -- (ra);
                \draw[arrow] (ra) -- (fp);
                \draw[arrow] (fp) -- (fb);
                \draw[arrow] (fb) -- (fbo);
                \draw[arrow] (fb) -- (display);
                \draw[arrow] (fs) -- (fp);
                \draw[arrow] (td) -- (fp);
                \draw[arrow] (gs) -- (pa);
                \draw[arrow] (ts) -- (pa);
                \draw[arrow] (vs) -- (vp);
                \draw[arrow] (fbo) -- (td);
                \draw[arrow] (gi) -- (input);
                \draw[arrow] ($(dt.north) + (0.0,0.05)$) -| ($(fb.south) - (0.2,0.0)$);
                \draw[arrow] (ab) -- (fb);
                \draw[arrow] ($(so.north) + (0.0,0.05)$) -| ($(fb.south) + (0.2,0.0)$);
            \end{scope}
            % why the below of= does not work with other nodes that are not a fixed one?
            \begin{scope}[below of=input,xshift=0.7cm,yshift=-0.9cm,scale=0.7]
                % input points
                \node[scale=0.35,circle,fill=red] at (0.00,0.00) {};
                \node[scale=0.35,circle,fill=red] at (0.60,0.25) {};
                \node[scale=0.35,circle,fill=red] at (0.25,0.75) {};
                \node[scale=0.35,circle,fill=red] at (1.15,0.95) {};
                \node[scale=0.35,circle,fill=red] at (1.05,0.45) {};
                \draw[thick,-{Latex[length=2.0mm]}] (-0.2,-0.2) -- (+1.25,-0.20);
                \draw[thick,-{Latex[length=2.0mm]}] (-0.2,-0.2) -- (-0.20,+1.25);
            \end{scope}
            \begin{scope}[below of=input,xshift=3.00cm,yshift=-0.9cm,scale=0.7]
                % transformed points
                \node[scale=0.35,circle,fill=blue] at (0.00,0.00) {};
                \node[scale=0.35,circle,fill=blue] at (0.60,0.25) {};
                \node[scale=0.35,circle,fill=blue] at (0.25,0.75) {};
                \node[scale=0.35,circle,fill=red] at (0.95,0.95) {};
                \node[scale=0.35,circle,fill=red] at (1.25,0.45) {};
                \draw[thick,-{Latex[length=2.0mm]}] (-0.2,-0.2) -- (+1.25,-0.20);
                \draw[thick,-{Latex[length=2.0mm]}] (-0.2,-0.2) -- (-0.20,+1.25);
            \end{scope}
            \begin{scope}[below of=input,xshift=5.75cm,yshift=-0.9cm,scale=0.7]
                % postprocessing points
                \draw[thick,-{Latex[length=2.0mm]}] (-0.2,-0.2) -- (+1.25,-0.20);
                \draw[thick,-{Latex[length=2.0mm]}] (-0.2,-0.2) -- (-0.20,+1.25);
            \end{scope}
            \begin{scope}[below of=input,xshift=6.75cm,yshift=-0.9cm,scale=0.7]
                % shape assembly
                \node[scale=0.35,circle,fill=blue] at (2.00,0.00) {};
                \node[scale=0.35,circle,fill=blue] at (2.60,0.25) {};
                \node[scale=0.35,circle,fill=blue] at (2.25,0.75) {};
                \node[scale=0.35,circle,fill=red] at (2.95,0.95) {};
                \node[scale=0.35,circle,fill=red] at (3.25,0.45) {};
                \draw[blue,line width=2pt](2.00,0.00) -- (2.60,0.25){};
                \draw[blue,line width=2pt](2.60,0.25) -- (2.25,0.75){};
                \draw[blue,line width=2pt](2.25,0.75) -- (2.00,0.00){};
                \draw[red,line width=2pt](2.95,0.95) -- (3.25,0.45){};
                \draw[thick,-{Latex[length=2.0mm]},xshift=2cm] (-0.2,-0.2) -- (+1.25,-0.20);
                \draw[thick,-{Latex[length=2.0mm]},xshift=2cm] (-0.2,-0.2) -- (-0.20,+1.25);
            \end{scope}
            \begin{scope}[below of=input,xshift=8.75cm,yshift=-0.9cm,scale=0.7]
                % grid of pixels
                \draw[step=5pt,black,xshift=2cm] (0,0) grid (1.59,1.24);
                \draw[thick,-{Latex[length=2.0mm]},xshift=2cm] (-0.2,-0.2) -- (+1.25,-0.20);
                \draw[thick,-{Latex[length=2.0mm]},xshift=2cm] (-0.2,-0.2) -- (-0.20,+1.25);
                \filldraw[fill=blue,xshift=2cm] (0*0.175,0*0.175) rectangle (1*0.175,1*0.175);
                \filldraw[fill=blue,xshift=2cm] (0*0.175,1*0.175) rectangle (1*0.175,2*0.175);
                \filldraw[fill=blue,xshift=2cm] (0*0.175,2*0.175) rectangle (1*0.175,3*0.175);
                \filldraw[fill=blue,xshift=2cm] (1*0.175,3*0.175) rectangle (2*0.175,4*0.175);
                \filldraw[fill=blue,xshift=2cm] (1*0.175,4*0.175) rectangle (2*0.175,5*0.175);
                \filldraw[fill=blue,xshift=2cm] (2*0.175,3*0.175) rectangle (3*0.175,4*0.175);
                \filldraw[fill=blue,xshift=2cm] (2*0.175,2*0.175) rectangle (3*0.175,3*0.175);
                \filldraw[fill=blue,xshift=2cm] (3*0.175,1*0.175) rectangle (4*0.175,2*0.175);
                \filldraw[fill=blue,xshift=2cm] (1*0.175,0*0.175) rectangle (2*0.175,1*0.175);
                \filldraw[fill=blue,xshift=2cm] (2*0.175,0*0.175) rectangle (3*0.175,1*0.175);
                \filldraw[fill=red,xshift=2cm] (5*0.175,5*0.175) rectangle (6*0.175,6*0.175);
                \filldraw[fill=red,xshift=2cm] (5*0.175,4*0.175) rectangle (6*0.175,5*0.175);
                \filldraw[fill=red,xshift=2cm] (6*0.175,3*0.175) rectangle (7*0.175,4*0.175);
                \filldraw[fill=red,xshift=2cm] (7*0.175,2*0.175) rectangle (8*0.175,3*0.175);
            \end{scope}
            \begin{scope}[below of=input,xshift=11.25cm,yshift=-0.9cm,scale=0.7]
                % grid of pixels with modified colors
                \draw[step=5pt,black,xshift=2cm] (0,0) grid (1.59,1.24);
                \draw[thick,-{Latex[length=2.0mm]},xshift=2cm] (-0.2,-0.2) -- (+1.25,-0.20);
                \draw[thick,-{Latex[length=2.0mm]},xshift=2cm] (-0.2,-0.2) -- (-0.20,+1.25);
                \filldraw[fill=blue!70!green,xshift=2cm] (0*0.175,0*0.175) rectangle (1*0.175,1*0.175);
                \filldraw[fill=blue!60!green,xshift=2cm] (0*0.175,1*0.175) rectangle (1*0.175,2*0.175);
                \filldraw[fill=blue!50!green,xshift=2cm] (0*0.175,2*0.175) rectangle (1*0.175,3*0.175);
                \filldraw[fill=blue!40!green,xshift=2cm] (1*0.175,3*0.175) rectangle (2*0.175,4*0.175);
                \filldraw[fill=blue!30!green,xshift=2cm] (1*0.175,4*0.175) rectangle (2*0.175,5*0.175);
                \filldraw[fill=blue!40!green,xshift=2cm] (2*0.175,3*0.175) rectangle (3*0.175,4*0.175);
                \filldraw[fill=blue!50!green,xshift=2cm] (2*0.175,2*0.175) rectangle (3*0.175,3*0.175);
                \filldraw[fill=blue!60!green,xshift=2cm] (3*0.175,1*0.175) rectangle (4*0.175,2*0.175);
                \filldraw[fill=blue!70!green,xshift=2cm] (1*0.175,0*0.175) rectangle (2*0.175,1*0.175);
                \filldraw[fill=blue!70!green,xshift=2cm] (2*0.175,0*0.175) rectangle (3*0.175,1*0.175);
                \filldraw[fill=red!80!white,xshift=2cm] (5*0.175,5*0.175) rectangle (6*0.175,6*0.175);
                \filldraw[fill=red!70!white,xshift=2cm] (5*0.175,4*0.175) rectangle (6*0.175,5*0.175);
                \filldraw[fill=red!60!white,xshift=2cm] (6*0.175,3*0.175) rectangle (7*0.175,4*0.175);
                \filldraw[fill=red!50!white,xshift=2cm] (7*0.175,2*0.175) rectangle (8*0.175,3*0.175);
            \end{scope}
        \end{tikzpicture}
    \end{customcenter}
    \begin{minipage}[t]{0.3125\textwidth}
        \colorfulsection{Vertex Processor}
        \vspace{-10pt}
        \begin{mdframed}[]
            \centering{Transforms vertex attributes\\ (position, color, normal, texture)}
        \end{mdframed}
        \vspace{-13pt}
        \begin{itemize}[leftmargin=*,label=$\blacksquare$]
            \setlength\itemsep{-2pt}
            \item Retrieves \textbf{vertex arrays from CPU} via Vertex Array Object (VAO) with\vspace{-4pt}
            \begin{itemize}[leftmargin=*,label=$\bullet$]
                \setlength\itemsep{0pt}
                \item $\substack{\text{Vertex Buffer Object}\\\text{VBO}}$ $\mapsto$ \textbf{geometry}
                \item $\substack{\text{Index Buffer Object}\\\text{IBO}}$ $\mapsto$ \textbf{topology}
            \end{itemize}
            \item Retrieves \textbf{shaders} uniforms
            \item \textbf{Execution is parallel} for each vertex
            \item Programmer has \textbf{no knowledge} of order execution
            \item Output \textbf{must} be in \textbf{normalize device coordinates} (aka \textit{clip coordinates})\vspace{-4pt}
            \begin{itemize}[leftmargin=*,label=$\bullet$]
                \setlength\itemsep{0pt}
                \item Vector in \textbf{homogeneous coordinates} with $\parenthesisA{x,y,z}$ in the range $\bracketA{-1,1}$ (viewing volume)
                \item Coordinates range can be achieved by \textit{model}$\parenthesisA{\mathbf{M}}$, \textit{view}$\parenthesisA{\mathbf{V}}$, and \textit{projection}$\parenthesisA{\mathbf{P}}$ transformations. 
            \end{itemize}
        \end{itemize}
    \end{minipage}
    \hspace{10pt}
    \begin{minipage}[t]{0.3125\textwidth}
        \colorfulsection{Vertex Post Processor}
        \vspace{-10pt}
        \begin{mdframed}[]
            \centering{Geometry tests for vertex data \\ (clipping, perspective division, etc.)}
        \end{mdframed}
        \vspace{-13pt}
        \begin{itemize}[leftmargin=*,label=$\blacksquare$]
            \setlength\itemsep{-2pt}
            \item All primitives in the viewing volume boundary are \textbf{broken into smaller primitives} that are inside the volume (\textbf{clipping})
            \item \textbf{Perspective division} by $w$
            \item Vertex is mapped into discrete screen coordinates (\textbf{viewport transformation}) according to the output framebuffer
        \end{itemize}
    \end{minipage}
    \hspace{10pt}
    \begin{minipage}[t]{0.3125\textwidth}
        \colorfulsection{Primitive Assembly}
        \vspace{-10pt}
        \begin{mdframed}[]
                \centering{Assembles data into primitives \\ (points, lines, triangles)}
        \end{mdframed}
        \vspace{-13pt}
        \begin{itemize}[leftmargin=*,label=$\blacksquare$]
            \setlength\itemsep{-2pt}
            \item Vertices are made primitives. \textbf{Typically triangles}.
            \item \textbf{Face culling} is optional\vspace{-4pt}
            \begin{itemize}[leftmargin=*,label=$\bullet$]
                \setlength\itemsep{0pt}
                \item Normal of rendering plane $\mathbf{k}$
                \item Normal of triangle $\mathbf{n} = \parenthesisA{\mathbf{p}_1 - \mathbf{p}_0}\times\parenthesisA{\mathbf{p}_2 - \mathbf{p}_0}$
                \item Dot product $\mathbf{n}\cdot\mathbf{k}$ provides direction (\textbf{back} or \textbf{front}) of triangle facing the camera
            \end{itemize}
        \end{itemize}
    \end{minipage}
    \begin{minipage}[t]{0.3125\textwidth}
        \colorfulsection{Rasterizer}
        \vspace{-10pt}
        \begin{mdframed}[]
                \centering{Maps primitives to pixel locations}
        \end{mdframed}
        \vspace{-13pt}
        \begin{itemize}[leftmargin=*,label=$\blacksquare$]
            \setlength\itemsep{-2pt}
            \item Primitives are rasterized in the \textbf{order given} by previous stage
            \item Result is a \textbf{fragment}, which are the set of properties to compute the final color of the pixel
            \item \textbf{Multisampling} is optional and if enabled creates a fragment per sample
        \end{itemize}
    \end{minipage}
    \hspace{10pt}
    \begin{minipage}[t]{0.3125\textwidth}
        \colorfulsection{Fragment Processor}
        \vspace{-10pt}
        \begin{mdframed}[]
                \centering{Decides appearance of each fragment \\ (lighting$|$texture mapping)}
        \end{mdframed}
        \vspace{-13pt}
        \begin{itemize}[leftmargin=*,label=$\blacksquare$]
            \setlength\itemsep{-2pt}
            \item \textbf{Execution is parallel} for each fragment$|$sample
            \item Programmer has \textbf{no knowledge} of order execution
            \item You can \textbf{discard} a fragment
            \item Output is \textbf{one color} per each output to the framebuffer expected \textbf{plus depth} and stencil values
            \item If \textit{fragment shader} is not supplied, color is \textit{undefined}
        \end{itemize}
    \end{minipage}
    \hspace{10pt}
    \begin{minipage}[t]{0.3125\textwidth}
        \colorfulsection{Per Fragment$|$Sample Operations}
        \vspace{-10pt}
        \begin{mdframed}[]
                \centering{Check if fragment$|$sample will become a pixel in framebuffer}
        \end{mdframed}
        \vspace{-13pt}
        \begin{itemize}[leftmargin=*,label=$\blacksquare$]
            \setlength\itemsep{-2pt}
            \item \textbf{Pixel ownership}: If pixel is not owned by OpenGL context test fails
            \item \textbf{Scissor}: Fails if pixel is outside a \textbf{user defined rectangle}
            \item \textbf{Depth}: Pixel depth is compared with one currently in the framebuffer, if less overwrites buffer and test passes, if not fragment is discarded
            \item \textbf{Blending}: Uses alpha value to perfom transparency
            \item \textbf{Writing Framebuffer}: You can disable certain channels to not be written into
        \end{itemize}
    \end{minipage}
\end{minipage}
\newpage
\begin{minipage}[t]{0.485\textwidth}
    \colorfulsection{Transformations}
    \begin{customcenter}[5pt]
        \begin{tikzpicture}
            % \draw[step=1cm,black,very thin] (0,0) grid (10,8);
            \node[draw,circle,minimum size=0.3cm,fill=babyblue,thick] (one) at (7,1.5) {\textbf{1}};
            \node[rounded corners,below of=one,draw,text width=4cm,text centered,xshift=-1.75cm] {Model Position \\ \textbf{Model Transformation}};
            \node[draw,circle,minimum size=0.3cm,fill=babyblue,thick] (two) at (1,4.495) {\textbf{2}};
            \node[rounded corners,above of=two,draw,text width=3.75cm,text centered,xshift=0.75cm] {Camera Position \\ \textbf{View Transformation}};
            \node[draw,circle,minimum size=0.3cm,fill=babyblue,thick] (three) at (1,6.5) {\textbf{3}};
            \node[rounded corners,above of=three,draw,text width=4.2cm,text centered,xshift=1cm] {Camera Lens \\ \textbf{Project Transformation}};
            \draw[thick] (0.5,1.5) -- (1.5,0.5);
            \draw[thick] (1.0,2.0) -- (2.0,1.0);
            \draw[thick] (0.5,1.5) -- (1.0,2.0);
            \draw[thick] (1.5,0.5) -- (2.0,1.0);
            \draw[thick] (0.5,1.0) -- (1.5,0.0);
            \draw[thick] (0.5,1.0) -- (0.5,1.5);
            \draw[thick] (1.5,0.0) -- (1.5,0.5);
            \draw[thick] (1.5,0.0) -- (2.0,0.5);
            \draw[thick] (2.0,0.5) -- (2.0,1.0);
            \draw[thick,dashed] (0.5,1.0) -- (1.0,1.5);
            \draw[thick,dashed] (1.0,1.5) -- (1.0,2.0);
            \draw[thick,dashed] (1.0,1.5) -- (2.0,0.5);            
            % \node[scale=0.5,circle,fill=black] at (5,7) {};
            % \node[scale=0.5,circle,fill=black] at (10,7) {};
            % \node[scale=0.5,circle,fill=black] at (5,3) {};
            % \node[scale=0.5,circle,fill=black] at (10,3) {};
            \draw[ultra thick] (5,7) -- (10,7);
            \draw[ultra thick] (5,7) -- (5,3);
            \draw[ultra thick] (5,3) -- (10,3);
            \draw[ultra thick] (10,3) -- (10,7);
            \draw[thick] (1.5,1.2) -- (5,7);
            \draw[thick] (1.5,1.2) -- (10,7);
            \draw[thick] (1.5,1.2) -- (10,3);
            \draw[thick,dashed] (1.5,1.2) -- (5,3);
            \node[isosceles triangle,draw,fill=red,minimum size=2.5cm,isosceles triangle apex angle=60,rotate=45] at (6.0,4.5) {};
            \node[isosceles triangle,draw,fill=red,minimum size=1.5cm,isosceles triangle apex angle=60,rotate=90] (tp) at (8.75,1.25) {};
            \node[xshift=-3pt,yshift=-3pt] at (tp.left corner) {$v_1$};
            \node[xshift=+5pt,yshift=-3pt] at (tp.right corner) {$v_2$};
            \node[xshift=+0pt,yshift=+3pt] at (tp.apex) {$v_3$};
        \end{tikzpicture}
    \end{customcenter}
    Using homogeneous coordinates is possible to express \textbf{translations}, \textbf{rotations}, \textbf{projections}, \textbf{scaling}, and \textbf{shearing} operations as linear transformations
    \begin{align*}
        & \overbrace{
            \begin{bmatrix}
                1 & 0 & 0 & x \\
                0 & 1 & 0 & y \\
                0 & 0 & 1 & z \\
                0 & 0 & 0 & 1 
            \end{bmatrix}
        }^{\text{Translation}}
        \overbrace{
            \begin{bmatrix}
                x & 0 & 0 & 0 \\
                0 & y & 0 & 0 \\
                0 & 0 & z & 0 \\
                0 & 0 & 0 & 1
            \end{bmatrix}
        }^{\text{Scaling}} \\
        & \overbrace{
            \underbrace{
                \begin{bmatrix}
                    1 & 0 & 0 & 0 \\
                    0 & c\theta & -s\theta & 0 \\
                    0 & s\theta & +c\theta & 0 \\
                    0 & 0 & 0 & 1
                \end{bmatrix}
            }_{\text{Rotation } X}
            \underbrace{
                \begin{bmatrix}
                    c\theta & 0 & -s\theta & 0 \\
                    0 & 1 & 0 & 0 \\
                    s\theta & 0 & +c\theta & 0 \\
                    0 & 0 & 0 & 1
                \end{bmatrix}
            }_{\text{Rotation } Y}
            \underbrace{
                \begin{bmatrix}
                    c\theta & -s\theta & 0 & 0 \\
                    s\theta & +c\theta & 0 & 0 \\
                    0 & 0 & 1 & 0 \\
                    0 & 0 & 0 & 1
                \end{bmatrix}
            }_{\text{Rotation } Z}
        }^{c\theta = \cos\theta,\;s\theta = \sin\theta}
    \end{align*}
\end{minipage}
\hspace{3pt}
\begin{minipage}[t]{0.485\textwidth}
    \vspace{-13.5pt}
    \begin{customcenter}[0pt]
        \begin{tikzpicture}
            \tikzstyle{arrow}=[thick,->,>=stealth]
            % \draw[step=1cm,black,very thin,opacity=1.0] (0,0) grid (10,-15);            
            \begin{scope}[xshift=-0.0cm]
                % boxes
                \node[rectangle,draw,minimum width=5.75cm,minimum height=9cm,line width=2pt] (descrcoordinates) at (5,-4.05){};
                \node[yshift=10pt] at (descrcoordinates.north){\colorfulsection{Graphics Coordinate System}};
                \node[rectangle,draw=blue,minimum width=4.25cm,minimum height=5cm,line width=1.5pt] (descrhomogeneous) at (5,-3.32){};
                \node[rotate=-90,yshift=5pt] at (descrhomogeneous.east){Homogeneous};
                % nodes
                \node (start) [rounded corners,fill=yellow!30,draw] at (5,0) {$\parenthesisA{x_o,y_o,z_o}$ Object};
                \node (hoco) [below of=start,rounded corners,fill=blue!30,draw,yshift=-0.35cm]{$\parenthesisA{x_m,y_m,z_m,1.0}$ Model};
                \node (hocw) [below of=hoco,rounded corners,fill=blue!30,draw,yshift=-0.35cm]{$\parenthesisA{x_w,y_w,z_w,1.0}$ World};
                \node (hoce) [below of=hocw,rounded corners,fill=blue!30,draw,yshift=-0.35cm]{$\parenthesisA{x_e,y_e,z_e,1.0}$ Eye};
                \node (hocc) [below of=hoce,rounded corners,fill=blue!30,draw,yshift=-0.35cm] {$\parenthesisA{x_c,y_c,z_c,w}$ Clip};
                \node (ndc) [below of=hocc,rounded corners,fill=yellow!30,draw,yshift=-0.35cm] {$\parenthesisA{x_{\text{ndc}},y_{\text{ndc}},z_{\text{ndc}}}$ $\substack{\text{Normalized Device}\\\parenthesisA{x,y}\in\bracketA{-1,1},z\in\bracketA{0,1}}$};
                \node (stop) [below of=ndc,rounded corners,fill=yellow!30,draw,text centered,yshift=-0.35cm] {$\parenthesisA{x,y}$ Window $|$ $z$ Depth};
                % connectors
                \draw[arrow] (start) -- node[anchor=east,yshift=+3pt] {Append $1.0$} (hoco);
                \draw[arrow] (hoco) -- node[anchor=west] {\textbf{Model}} node[anchor=east] {$\substack{\text{Scale}\\\text{Rotate}\\\text{Translate}}$} (hocw);
                \draw[arrow] (hocw) -- node[anchor=west] {\textbf{View}} node[anchor=east] {$\substack{\text{Orient}\\\text{Scene}}$} (hoce);
                \draw[arrow] (hoce) -- node[anchor=west] {\textbf{Projection}} node[anchor=east] {$\substack{\text{Perspective}\\\text{Ortographic}\\\text{Frustum}}$} (hocc);
                \draw[arrow] (hocc) -- node[anchor=east,yshift=-2.5pt] {Divide by $w$} (ndc);
                \draw[arrow] (ndc) -- node[anchor=west] {$\substack{\text{Clipping}\\\text{viewport}}$} (stop);
            \end{scope}
            % Projections descriptions
            \node[yshift=+1.50cm,left of=hoce,xshift=-2.45cm,rotate=90,draw,text width=2.5cm,text centered,thick] (descrpersp) {Further objects appear smaller};
            \node[left of=hoce,yshift=-0.425cm,xshift=-0.52cm] (perspnode) {};
            \draw[arrow] (descrpersp.west) |- (perspnode);
            \node[yshift=-1.75cm,left of=hocc,xshift=-2.45cm,rotate=90,draw,text width=2.5cm,text centered,thick] (descrortho) {Does not affect size of objects};
            \node[left of=hocc,yshift=+0.700cm,xshift=-0.52cm] (orthonode) {};
            \draw[arrow] (descrortho.east) |- (orthonode);
            % zfighting
            \node[xshift=2.0cm,draw,fill=red!40] (descrzfight) at (stop.east) {$z$-fighting};
            \node[yshift=10pt,above of=descrzfight,fill=orange!40,draw] (descrzfightreason) {$\substack{\text{Caused by}\\\text{floating point}\\\text{precision}}$};
            \node[yshift=11pt,above of=descrzfightreason,fill=green!40,draw] (solzfighting) {$\substack{\text{Possible solutions}\\\bullet\text{ Keep \texttt{near}$|$\texttt{far} close}\\\bullet\text{ Do not compress } z}$};
            \draw[arrow] (stop) -- (descrzfight);
            \draw[arrow] (descrzfight) -- (descrzfightreason);
            \draw[arrow] (descrzfightreason) -- (solzfighting);
            \node[rectangle,draw,minimum width=8cm,minimum height=4.5cm,line width=2pt] (vbobox) at (5.1,-12){};
            \node[yshift=10pt] at (vbobox.north) {\colorfulsection{VBO Arrangement}};
            \begin{scope}[xshift=1cm,yshift=1cm]
                \node[] at (2,-11.25) {\textbf{Separate VBO}};
                \node[rectangle,draw,fill=red!20,minimum width=3cm,minimum height=1cm,line width=1pt] () at (2,-12) {VERTICES};
                \node[rectangle,draw,fill=green!20,minimum width=3cm,minimum height=1cm,line width=1pt] () at (2,-13) {NORMALS};
                \node[rectangle,draw,fill=blue!20,minimum width=3cm,minimum height=1cm,line width=1pt] () at (2,-14) {TEXTURES};
                \node[] at (2,-14.75) {Dynamic Data};
                \node[] at (6,-11.25) {\textbf{Interleaved VBO}};
                \node[rectangle,draw,fill=red!20,minimum width=3cm,minimum height=0.75cm,line width=1pt,rotate=90] () at (4.75,-13) {$\text{VERTICES}_i$};
                \node[rectangle,draw,fill=green!20,minimum width=3cm,minimum height=0.75cm,line width=1pt,rotate=90] () at (5.50,-13) {$\text{NORMALS}_i$};
                \node[rectangle,draw,fill=blue!20,minimum width=3cm,minimum height=0.75cm,line width=1pt,rotate=90] () at (6.25,-13) {$\text{TEXTURES}_i$};
                \node[rectangle,left color=red!20,right color=green!20,minimum width=0.5cm,minimum height=3cm,line width=1pt] () at (6.85,-13) {};
                \node[rectangle,left color=green!20,right color=blue!20,minimum width=0.5cm,minimum height=3cm,line width=1pt] () at (7.35,-13) {};
                \node[rectangle,draw,minimum width=1cm,minimum height=3cm,line width=1pt] () at (7.115,-13) {$\dots$};
                \node[] at (6,-14.75) {Faster Access (Cache)};
            \end{scope}
        \end{tikzpicture}
    \end{customcenter}
\end{minipage}

\begin{minipage}[t]{0.485\textwidth}
    \vspace{5pt}
    \colorfulsection{Lighting}
    \begin{itemize}[leftmargin=*]
        \setlength\itemsep{0pt}
        \item \textit{Basic simplified} model is \textbf{Phong shading}. It consists of \textbf{3 terms}: \textit{ambient}, \textit{diffuse}, \textit{specular}
        \item \textbf{Ambient} term is a simplistic model of global illumination. It uses a \textbf{constant} color multiplied by a factor (\textit{strength})
        \begin{lstlisting}
            #version 330 core
            uniform vec3 objectcolor;
            uniform vec3 lightcolor;
            out vec4 fcolor;
            void main() 
            {
                float ambientstrength = 0.25;
                vec3 ambientterm =  (ambientstrength * lightcolor);
                vec3 ambientshading = (ambientterm * objectcolor);
                fcolor = vec4(ambientshading, 1.0);
            }
        \end{lstlisting}
        \item \textbf{Diffuse} term uses the normal to a surface $\parenthesisA{\mathbf{n}}$, and the light vector $\parenthesisA{\mathbf{l}}$, [direction from the \textbf{point on a surface} $\parenthesisA{\mathbf{p}_s}$ to the \textbf{light position} $\parenthesisA{\mathbf{p}_l}$] to deduce the contribution with
        \begin{align*}
            \mathbf{n}\cdot\mathbf{l} = \norm{\mathbf{l}}\norm{\mathbf{n}}\cos\theta
        \end{align*}
        \vspace{-15pt}
        \item Using normalized vectors the dot product becomes $\mathbf{\hat{n}}\cdot\mathbf{\hat{l}} = \cos\theta$, which is in the range $\bracketA{-1,+1}$
        \item Clamping to positive values the \textbf{diffuse term} is \underline{$\max(0, \mathbf{\hat{n}}\cdot\mathbf{\hat{l}})$}
    \end{itemize}
\end{minipage}
\hspace{10pt}
\begin{minipage}[t]{0.485\textwidth}
    \vspace{5pt}
    \begin{itemize}[leftmargin=*]
        \setlength\itemsep{0pt}
        \item Notice that we can choose $\mathbf{p}_l$ and $\mathbf{p}_s$ to be in \textbf{world coordinates}. There are different parts of the graphics pipeline to have both points in the same coordinate system but \textit{one standard} way is to have it in the \textit{fragment stage}
        \item $\mathbf{p}_s$ at the fragment stage can be computed from the vertex stage using the transformation \inlinecode{model * aposition}, for the \inlinecode{uniform mat4 model} variable and the \inlinecode{vec3 aposition} vertex attribute
        \item $\mathbf{p}_l$ can be given directly to the fragment shader as a \inlinecode{uniform vec3 lightposition} variable
        \item Given the \textit{selected implementation} for $\mathbf{p}_l$ and $\mathbf{p}_s$, the normal vector $n$ also requires to be in \quotes{\textit{world coordinates}}
        \item To do blindly \inlinecode{model * anormal}, for {\inlinecode{vec3 anormal}} a vertex attribute, \textbf{would be wrong} because\vspace{-4pt}
        \begin{enumerate}[leftmargin=*]
            \setlength\itemsep{0pt}
            \item A \textbf{normal represents a direction} and not a point
            \item Homogeneous coordinates for a direction has $w = 0$. This means that \textbf{translations} (last column in transformations $\mathbf{T}$) \textbf{do not affect} $\mathbf{n}$, so one can reduce $\mathbf{T}$ to $\mathbf{T}_{3\times 3}$
            \item \textbf{Non-uniform scaling} changes the direction of $\mathbf{n}$ making it not perpendicular anymore to the surface resulting in light distortion
        \end{enumerate}
        \vspace{-3pt}
        \item The solution is to use the \textbf{normal matrix}
        \begin{align*}
            \mathbf{N} \equiv \parenthesisA{\mathbf{T}_{3\times 3}^{-1}}^{\text{T}}
        \end{align*}
    \end{itemize}
\end{minipage}
\newpage
\begin{minipage}[t]{0.485\textwidth}
    \begin{itemize}[leftmargin=*]
        \setlength\itemsep{0pt}
        \item Follow the next diagram and shaders to do diffuse lighting
        \begin{customcenter}[0pt]
            \begin{tikzpicture}
                \draw[step=1cm,white,very thin,opacity=1.0] (-5,-1.5) grid (5,1);
                \draw[orange!60!black,thick,fill=orange!60!black] (-2.75,-1) -- (2,-1) -- (2.75,-0.5) -- (-2.0,-0.5) -- (-2.75,-1);
                \node[xshift=0.75cm,yshift=-0.1cm] at (2,-1) {Surface};
                \draw[-{Latex[length=2.0mm]},thick] (0,-0.75) -- node[anchor=east,xshift=-0.05cm,yshift=-0.05cm]{$\theta$} node[anchor=west] {$\mathbf{n}$} (0,0.5);
                \node[circle,draw=yellow!80!black,fill=yellow!80!black,minimum size=4pt,scale=1.0] (lp) at (-2,1){};
                \node[circle,draw=black,fill=black,minimum size=1pt,scale=0.35] (sp) at (0,-0.75){};
                \draw[black,rotate=90,thick,xshift=0.1cm] (0,0) arc (0:42:1);
                \node[left of=lp,xshift=0.3cm] {Light}; 
                \draw[-{Latex[length=2.0mm]},thick] (0,-0.75) -- node[anchor=west,yshift=0.6cm,xshift=-0.5cm]{$\mathbf{l} = \parenthesisA{\mathbf{p}_l - \mathbf{p}_s}$} (-2,1);
                \node[below of=lp,yshift=0.5cm] {$\mathbf{p}_l$};
                \node[right of=sp,xshift=-0.65cm] {$\mathbf{p}_s$};
            \end{tikzpicture}
        \end{customcenter}
        \begin{lstlisting}
            #version 330 core
            in vec3 aposition;
            in vec3 anormal;
            out vec3 vposition;
            out vec3 vnormal;
            uniform mat3 normalmatrix;
            uniform mat4 model;
            uniform mat4 viewprojection;
            void main() 
            {
                vnormal = (normalmatrix * anormal);
                vposition = model * vec4(aposition, 1.0);
                gl_Position = viewprojection * vposition;
            }
        \end{lstlisting}
        \begin{lstlisting}
            #version 330 core
            in vec3 vnormal;
            in vec3 vposition;
            uniform vec3 objectcolor;
            uniform vec3 lightposition;
            out vec4 fcolor;
            void main() 
            {
                vec3 n = normalize(vnormal);
                vec3 l = normalize(lighposition - vposition);
                vec3 diffuseterm =  (max(0, dot(n, l)) * lightcolor);
                vec3 diffuseshading = (diffuseterm * objectcolor);
                fcolor = vec4(diffuseshading, 1.0);
            }
        \end{lstlisting}
        \item \textbf{Specular} term is \textbf{viewer dependent}
        \begin{customcenter}[0pt]
            \begin{tikzpicture}
                \draw[step=1cm,white,very thin,opacity=1.0] (-5,-1.5) grid (5,1);
                \draw[cyan!60!black,thick,fill=cyan!60!black] (-2.75,-1) -- (2,-1) -- (2.75,-0.5) -- (-2.0,-0.5) -- (-2.75,-1);
                \node[xshift=0.75cm,yshift=-0.1cm] at (2,-1) {Surface};
                \draw[-{Latex[length=2.0mm]},thick,red!60] (0,-0.75) -- node[black,yshift=0.2cm]{$\mathbf{r}$} (2,1);
                \draw[-{Latex[length=2.0mm]},thick] (0,-0.75) -- node[anchor=west] {$\mathbf{n}$} (0,0.5);
                \node[circle,draw=yellow!80!black,fill=yellow!80!black,minimum size=4pt,scale=1.0] (lp) at (-2,1){};
                \node[circle,draw=black,fill=black,minimum size=1pt,scale=0.35] (sp) at (0,-0.75){};
                \node[left of=lp,xshift=0.3cm] {Light}; 
                \draw[-{Latex[length=2.0mm]},thick] (-2,1) -- node[anchor=east,yshift=-0.2cm,xshift=0.0cm]{$\mathbf{l} = \parenthesisA{\mathbf{p}_s - \mathbf{p}_l}$} (0,-0.75);
                \node[below of=lp,yshift=0.5cm] {$\mathbf{p}_l$};
                \node[right of=sp,xshift=-0.7cm,yshift=-0.15cm] {$\mathbf{p}_s$};
                \draw[-{Latex[length=2.0mm]},thick,blue!80] (0,-0.75) -- node[black,xshift=-0.1cm,yshift=0.325cm]{$\phi$} (3,0);
                \draw[xshift=3cm,rotate=25,thick] (0,+0.15) -- (0.25,0) node[xshift=0.65cm,rotate=20,yshift=0.2cm] {Viewer} node[yshift=-0.45cm,xshift=-0.8cm] {$\mathbf{v}$} node[yshift=-0.5cm] {$\mathbf{p}_v$} -- (0,-0.15);
                \draw[black,rotate=00,thick,xshift=1.2cm,yshift=-0.45cm] (0,0) arc (0:35:1);
            \end{tikzpicture}
        \end{customcenter}
        \item The specular term is represented by $\phi$ and using the dot product trick again it can be deduced as
        \begin{align*}
            (\max(0,\cos\phi))^{\gamma} = (\max(0,\mathbf{\hat{v}}\cdot\mathbf{\hat{r}}))^{\gamma}
        \end{align*}
        \item $\mathbf{\hat{v}}$ is the normalized \textbf{viewer direction}. It is computed from the viewer position $\mathbf{p}_v$ and the surface position $\mathbf{p}_s$
        \begin{align*}
            \mathbf{v} = (\mathbf{p}_v - \mathbf{p}_s)
        \end{align*}
        \item $\mathbf{\hat{r}}$ is the normalized \textbf{reflection direction}. The computation for the reflection follows from computing the rejection of $\mathbf{\hat{l}}$ from $\mathbf{\hat{n}}$
        \begin{align*}
            \mathbf{\hat{r}} = \mathbf{\hat{l}} - 2\mathbf{\hat{l}}_{||\mathbf{\hat{n}}} \equiv \mathbf{\hat{l}} - 2(\mathbf{\hat{l}}\cdot\mathbf{\hat{n}})\mathbf{\hat{n}}
        \end{align*}
        \item Notice how for the specular term $\mathbf{l}$ is the vector \textbf{from the light to the surface} as opposed to \textit{from the surface to the light} in the diffuse term 
        \item The \textbf{parameter $\gamma$} is a \textit{power curve} value that makes the specular light to be \textbf{brightest} when $\mathbf{v}$ and $\mathbf{r}$ are \textbf{parallel} while decaying when moving the viewer from the reflection
    \end{itemize}
\end{minipage}
\hspace{10pt}
\begin{minipage}[t]{0.485\textwidth}
    \begin{itemize}[leftmargin=*]
        \setlength\itemsep{0pt}
        \item The reflection direction be computed directly in a shader via \inlinecode{reflect()} function
        \begin{lstlisting}
            #version 330 core
            in vec3 vnormal;
            in vec3 vposition;
            in vec3 veye;
            uniform vec3 objectcolor;
            uniform vec3 lightposition;
            uniform float gamma;
            out vec4 fcolor;
            void main() 
            {
                vec3 l = normalize(vposition - lightposition);
                vec3 n = normalize(vnormal);
                vec3 v = normalize(veye - vposition);
                vec3 r = reflect(l, n);
                vec3 specularterm =  pow(max(0, dot(v, r)), gamma);
                vec3 color = (lightcolor * objectcolor);
                vec3 specularshading = (specularterm * color);
                fcolor = vec4(specularshading, 1.0);
            }
        \end{lstlisting}
        \item All previous calculations can be done in \textit{view space}. The advantage of that is having the viewer position for free (as $(0,0,0)$). If using such space, relevant vectors need to be multiplied by the \textbf{view matrix} as well. Don't forget that in such case the normal matrix needs to be recomputed as well.
        \item Previous implementation does lighting in the fragment shader. If computations are made in the vertex shader (you gain speed with fewer vertices than pixels but you lose smoothness as the fragments would be interpolated from the values of vertices) is called \textbf{Gouraud shading}
    \end{itemize}
\end{minipage}