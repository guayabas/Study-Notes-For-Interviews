%%%%%%%%%% Page 1 - Col 1 %%%%%%%%%%
\newpage
\colorfulheader{computer graphics}

\begin{minipage}[t]{0.485\textwidth}
    \colorfulsection{Basics}
    \begin{itemize}[leftmargin=*]
        \setlength\itemsep{0pt}
        \item A \textbf{pixel} (picture element) represents a single \quotes{\textit{point}} in a raster image. Its intensity has typically 3 components: \textbf{red}, \textbf{green}, and \textbf{blue}.
        \item Graphics can be \textbf{raster}-type, where an image is an array of pixels, or \textbf{vector}-type consisting of encoding information about shape and color of an image.
        \item A \textbf{primitive} represents a \textbf{basic unit} to create more complex objects. For example, sprites, text characters, or geometric shapes (\textbf{point}, \textbf{line}, and \textbf{triangle}) are primitives.
        \item The term \textbf{rendering} can be explained as \textbf{generating a 2D image from a 3D scene} by means of a computer. The scene can contain information about the geometry, camera, texture mapping, lighting model, shading effects, etc.
        \item Rendering can be implemented by 3 methods
        \vspace{-5pt}
        \begin{enumerate}[leftmargin=*]
            \setlength\itemsep{0pt}
            \item \textbf{scan-line} (also know as rasterization) algorithms
            \item \textbf{ray-tracing} techniques
            \item via solving the \textbf{rendering equation}
        \end{enumerate}
        \item Methods 2 and 3 are realistic rendering that simulates 2 relevant aspects of light: [1] \textbf{transport} (how much light passes from one place to another), and [2] \textbf{scattering} (how surfaces interact with light). Method 1 uses \textbf{shading} (approximation of local light), such as the \textbf{Phong illumination model}, to decide the coloring of the rasterized image.
        \item One can add detailed color or patterns into a surface of a 3D model via \textbf{texture mapping}. Polygonal surfaces would require the addition of \textbf{texture coordinates} (also know as UV mapping) while parametric surfaces (such as non-uniform rational b-spline -NURB-) would have an intrinsic texture coordinate.
        \item To add realism in shaders, there are various texture techniques (\textbf{mappings}) such as: normal, bump, parallax, specular, shadow, environmental, etc.
        \item In raster images, an \textit{artifact} called \textbf{aliasing} is presented by introducing low frequency information that is visually shown as noise in geometric edges or boundaries of textures. Some possible solutions are \textbf{supersampling}, \textbf{mipmapping}, or \textbf{texture filtering} (nearest-neighbor, bilinear, trilinear, anisotropic, etc.)
    \end{itemize}
    \colorfulsection{Analytical Geometry}
    \begin{itemize}[leftmargin=*]
        \setlength\itemsep{0pt}
        \item With vectors $\mathbf{u}$ and $\mathbf{v}$, \textit{relevant relations} between them are
        \begin{align*}
            & \parenthesisA{\text{\underline{Dot Product}}}\hspace{5pt}\mathbf{u}\cdot\mathbf{v} = \norm{\mathbf{u}}\hspace{1pt}\norm{\mathbf{v}}\cos\theta \\
            & \parenthesisA{\underline{\text{Projection of }\mathbf{u}\text{ on }\mathbf{v}}}\hspace{5pt}\text{Proj}_{\mathbf{v}}\mathbf{u} = \parenthesisA{\frac{\mathbf{u}\cdot\mathbf{v}}{\mathbf{v}\cdot\mathbf{v}}}\mathbf{v}\\
            & \parenthesisA{\text{\underline{Cross Product Orthogonality}}}\hspace{5pt}\parenthesisA{\mathbf{u}\times\mathbf{v}}\cdot\mathbf{u} = \parenthesisA{\mathbf{u}\times\mathbf{v}}\cdot\mathbf{v} = 0 \\
            & \parenthesisA{\text{\underline{Cross Product Norm}}}\hspace{5pt}\norm{\mathbf{u}\times\mathbf{v}} = \norm{\mathbf{u}}\hspace{1pt}\norm{\mathbf{v}}\sin\theta
        \end{align*}
        \item Point to line distance
        \begin{customcenter}[0pt]
            \begin{tikzpicture}
                \begin{scope}[xshift=0cm]
                    % \draw[step=1cm,black,very thin] (0,0) grid (8.75,2.5);
                    \draw[dashed,very thick,red] (4.5,2.0) -- (4.725,1.175);
                    \draw[-{Latex[length=2.5mm]}] (0,0) -- (4.5,2.0);
                    \node[] at (1.35,1.1) {$\mathbf{u} = \mathbf{p} - \mathbf{a}$};
                    \node[circle,draw=orange,fill=orange,minimum size=1pt,scale=0.5] (p) at (4.5,2.0){};
                    \node[above of=p,yshift=-0.65cm,xshift=0.0cm] {$\mathbf{p}$};
                    \draw[-{Latex[length=2.5mm]}] (0,0) -- (4,1);
                    \draw[thick,line width=1.5pt,opacity=0.5] (-1,-0.25) -- (5,1.25);
                    \node[circle,draw=blue,fill=blue,minimum size=1pt,scale=0.5] (a) at (0,0){};
                    \node[below of=a,yshift=0.7cm,xshift=0.025cm] {$\mathbf{a}$};
                    \node[circle,draw=blue,fill=blue,minimum size=1pt,scale=0.5] (b) at (4,1){};
                    \node[below of=b,yshift=0.7cm,xshift=0.025cm] {$\mathbf{b}$};
                    \node[] at (2.25,0.175) {$\mathbf{v} = \mathbf{b} - \mathbf{a}$};
                    \node[circle,draw=orange,fill=orange,minimum size=1pt,scale=0.5] (q) at (4.725,1.175){};
                    \node[below of=q,yshift=0.65cm,xshift=0.0cm] {$\mathbf{q}$};
                    \node[draw] at (7.0,1.5) {$\mathbf{q} = \mathbf{a} + \text{Proj}_{\mathbf{v}}\mathbf{u}$};
                    \node[draw,fill=black!25] at (7.0,0.5) {$d\parenthesisA{\mathbf{p},\mathrm{\mathbf{L}}} = d\parenthesisA{\mathbf{p},\mathbf{q}}$};
                    \node[] at (-0.5,0.15) {$\mathrm{\mathbf{L}}$};
                \end{scope}
            \end{tikzpicture}
        \end{customcenter}
    \end{itemize}
\end{minipage}
%%%%%%%%%%%%%%%%%%%%%%%%%%%%%%%%%%%%
\hspace{10pt}
%%%%%%%%%% Page 1 - Col 2 %%%%%%%%%%
\begin{minipage}[t]{0.485\textwidth}
    \colorfulsection{Curves}
    \emptyline

    Can be defined as a \textbf{trajectory of a moving point}. It can be represented as follow
    \begin{customcenter}[5pt]
        \begin{tabular}{|l|l|l|l|}
            \hline
            & \textbf{Explicit} & \textbf{Implicit} & \textbf{Parametric} \\
            & $y = f\parenthesisA{x}$ & $f\parenthesisA{x, y} = 0$ & $q\parenthesisA{t} = \bracketA{x\parenthesisA{t}, y\parenthesisA{t}}$ \\
            \hline
            Circle & $\sqrt{r^2 - x^2}$ & $x^2 + y^2 - r^2$ & $r\bracketA{\sin\parenthesisA{t}, \cos\parenthesisA{t}}$ \\
            \hline
            Parabola & $x^2$ & $y - x^2$ & $\bracketA{t, t^2}$ \\
            \hline
        \end{tabular}
    \end{customcenter}
    The \textbf{parametrization of a line} given 2 points, $\mathcal{A}$ and $\mathcal{B}$, can be thought as an initial \textbf{point} and a \textbf{direction} as
    \begin{align*}
        \text{Line}\parenthesisA{t} &= \mathcal{A} + t\parenthesisA{\mathcal{B - A}},\;t\in\mathbb{R} \\
        \text{Line}\parenthesisA{t} &= \parenthesisA{1 - t}\mathcal{A} + t\mathcal{B} \\
        \text{Line}\parenthesisA{t} &= \bracketA{\parenthesisA{1 - t}x_0 + ty_0, \dots, \parenthesisA{1 - t}x_n + ty_n}
    \end{align*}
    The \textbf{derivative} of a curve is the \textbf{tangent vector} at a given point.\emptyline

    Given the \textbf{weights} $a_i$ and the \textbf{basis functions} $b_i$, for $i\in\bracketA{0, n}$, a \textbf{polynomial curve} can be represented as
    \begin{align*}
        q\parenthesisA{t} = a_0b_0\parenthesisA{t} + \dots + a_nb_n\parenthesisA{t} = \sum_{k = 0}^{n} a_kb_k\parenthesisA{t}
    \end{align*}
    For example, a line would be $b_0 = \parenthesisA{1 - t}$, $b_1 = t$, and $b_{i \geq 2} = 0.$\emptyline

    \textbf{Relevant} curves are \textbf{degree 3} (also know as \textbf{cubics}) given that 3 is the lowest degree that can represented an \textbf{S-shape} (shape with an inflection point). Since the cubic curve
    \begin{align*}
        q\parenthesisA{t} = \bracketA{f_0\parenthesisA{t}, f_1\parenthesisA{t}, f_2\parenthesisA{t}},\;f_u\parenthesisA{t} = x_u + y_ut + z_ut^2 + w_ut^3
    \end{align*}
    suffers from too many coefficients to define a shape, a solution is to use \textbf{control points}.
    \begin{customcenter}[5pt]
        \begin{tabular}{|l|c|c|c|c|}
            \hline
            \diagbox[width=2.1cm]{Schema}{Basis} & $b_0$ & $b_1$ & $b_2$ & $b_3$ \\
            \hline
            \textbf{Bezier} & $\parenthesisA{1 - t}^3$ & $3t\parenthesisA{1 - t}^2$ & $3t^2\parenthesisA{1 - t}$ & $t^3$ \\
            \hline
            \textbf{Hermite} & $2t^3 - 3t^2 + 1$ & $-2t^3 + 3t^2$ & $t^3 - 2t^2 + t$ & $t^3 - t^2$ \\            
            \hline
        \end{tabular}
    \end{customcenter}
    \colorfulsection{Ray Tracing}
    \emptyline
    
    A color in the computer needs the discretization mapping
    \begin{align*}
    f = \left[0, 1\right)\mapsto\bracketA{0, \dots, 255} = i
    \end{align*}
    Due to non-linearity on screens one might do a \textbf{gamma correction}, $i = \text{\texttt{int}}\parenthesisA{256 * f^{\gamma}}$\emptyline

    Given the ray $\mathcal{R} = \mathbf{a}_r + t\mathbf{u},\; t\in\left[0,\infty\right)$ and an implicit surface $f\parenthesisA{\mathbf{p}} = 0$ closed solutions for the \textbf{intersection surface-ray} are
    \begin{customcenter}[5pt]
        \begin{tabular}{|l|c|}
            \hline
            Surface & \textbf{Intersection}\\
            \hline
            \textbf{Plane} & \multirow{2}{*}{$\frac{\parenthesisA{\mathbf{a}_p - \mathbf{a}_r}\cdot\mathbf{n}}{\mathbf{u}\cdot\mathbf{n}}$} \\
            $\parenthesisA{\mathbf{p} - \mathbf{a}_p}\cdot\mathbf{n} = 0$ & \\
            \hline
            \textbf{Sphere} & \multirow{2}{*}{$\frac{-\mathbf{u}\cdot\mathbf{b}\pm\bracketA{\parenthesisA{\mathbf{u}\cdot\mathbf{c}}^2 - \norm{\mathbf{u}}^2\parenthesisA{\norm{\mathbf{c}}^2 - r^2}}}{\norm{\mathbf{u}}^2}^{\frac{1}{2}}\begin{aligned}&\mathbf{b} = \parenthesisA{\mathbf{a}_r-\mathbf{u}}\\&\mathbf{c} = \parenthesisA{\mathbf{a}_r - \mathbf{a}_c}\end{aligned}$}\\
            $\norm{\mathbf{p} - \mathbf{a}_c}^2 - r^2 = 0$ &  \\[3pt]
            \hline
            Triangle & \\
            \hline
        \end{tabular}
    \end{customcenter}
\end{minipage}

