%%%%%%%%%% Page 1 - Col 1 %%%%%%%%%%
\newpage
\colorfulheader{computer graphics}

\begin{minipage}[t]{0.485\textwidth}
    \colorfulsection{Basics}
    \begin{itemize}[leftmargin=*]
        \setlength\itemsep{0pt}
        \item A \textbf{pixel} (picture element) represents a single \quotes{\textit{point}} in a raster image. Its intensity has typically 3 components: \textbf{red}, \textbf{green}, and \textbf{blue}.
        \item Graphics can be \textbf{raster}-type, where an image is an array of pixels, or \textbf{vector}-type consisting of encoding information about shape and color of an image.
        \item A \textbf{primitive} represents a \textbf{basic unit} to create more complex objects. For example, sprites, text characters, or geometric shapes (\textbf{point}, \textbf{line}, and \textbf{triangle}) are primitives.
        \item The term \textbf{rendering} can be explained as \textbf{generating a 2D image from a 3D scene} by means of a computer. The scene can contain information about the geometry, camera, texture mapping, lighting model, shading effects, etc.
        \item Rendering can be implemented by 3 methods
        \vspace{-5pt}
        \begin{enumerate}[leftmargin=*]
            \setlength\itemsep{0pt}
            \item \textbf{scan-line} (also know as rasterization) algorithms
            \item \textbf{ray-tracing} techniques
            \item via solving the \textbf{rendering equation}
        \end{enumerate}
        \item Methods 2 and 3 are realistic rendering that simulates 2 relevant aspects of light: [1] \textbf{transport} (how much light passes from one place to another), and [2] \textbf{scattering} (how surfaces interact with light). Method 1 uses \textbf{shading} (approximation of local light), such as the \textbf{Phong illumination model}, to decide the coloring of the rasterized image.
        \item One can add detailed color or patterns into a surface of a 3D model via \textbf{texture mapping}. Polygonal surfaces would require the addition of \textbf{texture coordinates} (also know as UV mapping) while parametric surfaces (such as non-uniform rational b-spline -NURB-) would have an intrinsic texture coordinate.
        \item To add realism in shaders, there are various texture techniques (\textbf{mappings}) such as: normal, bump, parallax, specular, shadow, environmental, etc.
        \item In raster images, an \textit{artifact} called \textbf{aliasing} is presented by introducing low frequency information that is visually shown as noise in geometric edges or boundaries of textures. Some possible solutions are \textbf{supersampling}, \textbf{mipmapping}, or \textbf{texture filtering} (nearest-neighbor, bilinear, trilinear, anisotropic, etc.)
    \end{itemize}
    \vspace{-3pt}
    \colorfulsection{Analytical Geometry}
    \begin{itemize}[leftmargin=*]
        \setlength\itemsep{0pt}
        \item With vectors $\mathbf{u}$ and $\mathbf{v}$, \textit{relevant relations} between them are
        \begin{align*}
            & \parenthesisA{\text{\underline{Dot Product}}}\hspace{5pt}\mathbf{u}\cdot\mathbf{v} = \norm{\mathbf{u}}\hspace{1pt}\norm{\mathbf{v}}\cos\theta \\
            & \parenthesisA{\underline{\text{\textit{Projection} of }\mathbf{u}\text{ onto }\mathbf{v}}}\hspace{5pt}\mathbf{u}_{||\mathbf{v}}\equiv\text{Proj}_{\mathbf{v}}\mathbf{u} = \parenthesisA{\frac{\mathbf{u}\cdot\mathbf{v}}{\mathbf{v}\cdot\mathbf{v}}}\mathbf{v}\\
            & \parenthesisA{\underline{\text{\textit{Rejection} of }\mathbf{u}\text{ from }\mathbf{v}}}\hspace{5pt}\mathbf{u}_{\perp\mathbf{v}}\equiv\text{Rej}_{\mathbf{v}}\mathbf{u}=\mathbf{u} - \mathbf{u}_{||\mathbf{v}}\\
            & \parenthesisA{\text{\underline{Cross Product Orthogonality}}}\hspace{5pt}\parenthesisA{\mathbf{u}\times\mathbf{v}}\cdot\mathbf{u} = \parenthesisA{\mathbf{u}\times\mathbf{v}}\cdot\mathbf{v} = 0 \\
            & \parenthesisA{\text{\underline{Cross Product Norm}}}\hspace{5pt}\norm{\mathbf{u}\times\mathbf{v}} = \norm{\mathbf{u}}\hspace{1pt}\norm{\mathbf{v}}\sin\theta
        \end{align*}
        \item Point to line distance
        \begin{customcenter}[0pt]
            \begin{tikzpicture}
                \begin{scope}[xshift=0cm]
                    % \draw[step=1cm,black,very thin] (0,0) grid (8.75,2.5);
                    \draw[dashed,very thick,red] (4.5,2.0) -- (4.725,1.175);
                    \draw[-{Latex[length=2.5mm]}] (0,0) -- (4.5,2.0);
                    \node[] at (1.35,1.1) {$\mathbf{u} = \mathbf{p} - \mathbf{a}$};
                    \node[circle,draw=orange,fill=orange,minimum size=1pt,scale=0.5] (p) at (4.5,2.0){};
                    \node[above of=p,yshift=-0.65cm,xshift=0.0cm] {$\mathbf{p}$};
                    \draw[-{Latex[length=2.5mm]}] (0,0) -- (4,1);
                    \draw[thick,line width=1.5pt,opacity=0.5] (-1,-0.25) -- (5,1.25);
                    \node[circle,draw=blue,fill=blue,minimum size=1pt,scale=0.5] (a) at (0,0){};
                    \node[below of=a,yshift=0.7cm,xshift=0.025cm] {$\mathbf{a}$};
                    \node[circle,draw=blue,fill=blue,minimum size=1pt,scale=0.5] (b) at (4,1){};
                    \node[below of=b,yshift=0.7cm,xshift=0.025cm] {$\mathbf{b}$};
                    \node[] at (2.25,0.175) {$\mathbf{v} = \mathbf{b} - \mathbf{a}$};
                    \node[circle,draw=orange,fill=orange,minimum size=1pt,scale=0.5] (q) at (4.725,1.175){};
                    \node[below of=q,yshift=0.65cm,xshift=0.0cm] {$\mathbf{q}$};
                    \node[draw] at (7.0,1.5) {$\mathbf{q} = \mathbf{a} + \text{Proj}_{\mathbf{v}}\mathbf{u}$};
                    \node[draw,fill=black!25,thick] at (7.0,0.5) {$d\parenthesisA{\mathbf{p},\mathrm{\mathbf{L}}} = d\parenthesisA{\mathbf{p},\mathbf{q}}$};
                    \node[] at (-0.5,0.15) {$\mathrm{\mathbf{L}}$};
                \end{scope}
            \end{tikzpicture}
        \end{customcenter}
    \end{itemize}
\end{minipage}
%%%%%%%%%%%%%%%%%%%%%%%%%%%%%%%%%%%%
\hspace{10pt}
%%%%%%%%%% Page 1 - Col 2 %%%%%%%%%%
\begin{minipage}[t]{0.485\textwidth}
    \begin{itemize}[leftmargin=*]
        \setlength\itemsep{0pt}
        \item Distance between two lines\\
        In $\mathbb{R}^3$ \textbf{two lines} that do not lie in the same plane are called \textbf{skewed}.
        \begin{customcenter}[0pt]
            \begin{tikzpicture}
                \draw[step=1cm,white,very thin] (-5,-1.5) grid (4.5,1.5);
                \begin{scope}[xshift=-1cm]
                    \draw[dashed,thick] (-2,0.575) -- node[anchor=east] {$d\parenthesisA{\mathcal{\mathbf{L}}_1,\mathcal{\mathbf{L}}_2}$} (-2,-0.55);
                    \draw[{Latex[length=2.5mm]}-{Latex[length=2.5mm]},green!50!black] (-4,+1) -- (2,-0.33);
                    \node[circle,draw=green!50!black,fill=green!50!black,minimum size=1pt,scale=0.5] (p1) at (-2,+0.56){};
                    \node[above of=p1,yshift=-0.7cm,xshift=0.65cm]{$\mathbf{p}_1 + t_1\mathbf{v}_1$};
                    \draw[{Latex[length=2.5mm]}-{Latex[length=2.5mm]},blue!80!black] (-3.5,-1) -- (2,+0.7);
                    \node[circle,draw=blue!80!black,fill=blue!80!black,minimum size=1pt,scale=0.5] (p2) at (-2,-0.54){};
                    \node[below of=p2,yshift=+0.7cm,xshift=0.65cm]{$\mathbf{p}_2 + t_2\mathbf{v}_2$};
                    \draw[] (-2.0,+0.35) -- (-1.8,+0.32);
                    \draw[] (-1.8,+0.32) -- (-1.8,+0.52);
                    \draw[] (-2.0,-0.35) -- (-1.8,-0.29);
                    \draw[] (-1.8,-0.29) -- (-1.8,-0.48);
                    \node[] at (-3.75,+1.25) {$\mathcal{\mathbf{L}}_1$};
                    \node[] at (-3.25,-0.65) {$\mathcal{\mathbf{L}}_2$};
                \end{scope}
                \begin{scope}
                    \node[] at (2.85,1.0) {$\parenthesisA{\mathcal{\mathbf{L}}_2 - \mathcal{\mathbf{L}}_1}\cdot\mathbf{v}_1 = 0$};
                    \node[] at (2.85,0.5) {$\parenthesisA{\mathcal{\mathbf{L}}_2 - \mathcal{\mathbf{L}}_1}\cdot\mathbf{v}_2 = 0$};
                    \draw[-implies,double equal sign distance,thick] (2.75,0.25) -- (2.75,-0.05);
                    \node[] at (2.85,-0.275) {$\alpha = \mathbf{v}_1\cdot\mathbf{v}_2$};
                    \node[draw,fill=black!25,thick] at (1.70,-1.0) {$\begin{bmatrix} \norm{\mathbf{v}_1}^2 & -\alpha \\ \alpha & -\norm{\mathbf{v}_2}^2\end{bmatrix}\hspace{-4pt}\begin{bmatrix}t_1 \\ t_2\end{bmatrix}\hspace{-3pt}=\hspace{-3pt}\begin{bmatrix}\parenthesisA{\mathbf{p}_2 - \mathbf{p}_1}\cdot\mathbf{v}_1  \\ \parenthesisA{\mathbf{p}_2 - \mathbf{p}_1}\cdot\mathbf{v}_2\end{bmatrix}$};
                \end{scope}
            \end{tikzpicture}
        \end{customcenter}
        \item Point to plane distance
        \begin{customcenter}[0pt]
            \begin{tikzpicture}
                \draw[step=1cm,white,very thin] (-5,-2.5) grid (5,2.5);
                \begin{scope}
                    \node[circle,draw=black,fill=black,minimum size=1pt,scale=0.5] (o) at (-4,-2){};
                    \node[below of=o,yshift=+0.80cm,xshift=0.1cm] {$\mathbf{o}$};
                    \node[above of=o,yshift=-0.55cm,xshift=0.3cm] {$-\mathbf{p}$};
                    \draw[-{Latex[length=2.5mm]}] (-4.0,-2.0) -- (-3.0, -2.0);
                    \node[] at (-3,-2.25) {$x$};
                    \draw[-{Latex[length=2.5mm]}] (-4.0,-2.0) -- (-4.0, -1.0);
                    \node[] at (-4,-0.8) {$y$};
                    \draw[-{Latex[length=2.5mm]}] (-4.0,-2.0) -- (-4.5, -2.5);
                    \node[] at (-4.65,-2.5) {$z$};
                \end{scope}
                \begin{scope}
                    \draw[dashed] (-0.35,-2.5) -- (-0.35,2.5);
                    \draw[-{Latex[length=2.5mm]},thick] (-0.35,0.0) -- (-4.00,-2.0);
                    \draw[fill=red!50,thick,yshift=0.25cm] (-3.5,-1) -- (-1,+1.0) -- (+3,+0.5) -- (+0.5,-1.5) -- (-3.5,-1);
                    \draw[-{Latex[length=2.5mm]},thick] (-0.35,0) -- node[anchor=east] {$\mathbf{\hat{n}}$} (-0.35,1.0);
                    \draw[-{Latex[length=2.5mm]},thick] (-0.35,0) -- node[anchor=east,yshift=1.0cm] {$\mathbf{n}$} (-0.35,2.0);
                    \node[fill=black!25,minimum width=1.45cm, minimum height=0.57cm] at (-1.195,1.45) {};
                    \draw[-{Latex[length=2.5mm]},thick,blue] (-0.35,0) -- node[draw,anchor=east,black,yshift=0.7cm,xshift=-0.1cm] {$d = \mathbf{q}_{||\mathbf{\hat{n}}}$} (-0.35,1.5);
                    \draw[-{Latex[length=2.5mm]},thick,orange!50!black] (-0.35,0) -- node[anchor=east,black,yshift=-0.7cm] {$\mathbf{-p}_{||\mathbf{\hat{n}}}$} (-0.35,-2.0);
                    \node[circle,draw=black,fill=black,minimum size=1pt,scale=0.5] (p) at (-0.35,0.0){};
                    \node[below of=p,yshift=0.70cm,xshift=-0.15cm] {$\mathbf{p}$};
                    \node[circle,draw=black,fill=black,minimum size=1pt,scale=0.5] (q) at (+0.75,1.5){};
                    \node[right of=q,yshift=0.20cm,xshift=-0.90cm] {$\mathbf{q}$};
                    \draw[-{Latex[length=2.5mm]},thick] (-0.35,0.0) -- node[anchor=west,xshift=-0.1cm,yshift=-0.25cm] {$\mathbf{q - p}$} (+0.75,+1.5);
                    \draw[dash dot,thick] (-0.35,0.0) -- (-2.00,-0.905);
                    \draw[dashed] (0.75,1.5) -- (-0.35,1.5);
                    \draw[dashed] (-4.0,-2.0) -- (-0.35,-2.0);
                    \node[xshift=2.1cm] at (0.5,-1.5) {$\mathcal{\mathbf{P}}\equiv\parenthesisA{\mathbf{q} - \mathbf{p}}\cdot\mathbf{n} = 0,\;\mathbf{q}\in\mathbb{R}^3$};
                \end{scope}
            \end{tikzpicture}
        \end{customcenter}
    \end{itemize}
    \colorfulsection{Curves}
    \begin{itemize}[leftmargin=*]
        \setlength\itemsep{0pt}
        \item Can be defined as a \textbf{trajectory of a moving point}. It can be represented as follow
        \begin{customcenter}[0pt]
            \begin{tabular}{|l|l|l|l|}
                \hline
                & \textbf{Explicit} & \textbf{Implicit} & \textbf{Parametric} \\
                & $y = f\parenthesisA{x}$ & $f\parenthesisA{x, y} = 0$ & $q\parenthesisA{t} = \bracketA{x\parenthesisA{t}, y\parenthesisA{t}}$ \\
                \hline
                Circle & $\sqrt{r^2 - x^2}$ & $x^2 + y^2 - r^2$ & $r\bracketA{\sin\parenthesisA{t}, \cos\parenthesisA{t}}$ \\
                \hline
                Parabola & $x^2$ & $y - x^2$ & $\bracketA{t, t^2}$ \\
                \hline
            \end{tabular}
        \end{customcenter}
        \item The \textbf{parametrization of a line} given 2 points, $\mathcal{A}$ and $\mathcal{B}$, can be thought as an initial \textbf{point} and a \textbf{direction} as
        \begin{align*}
            \text{Line}\parenthesisA{t} &= \mathcal{A} + t\parenthesisA{\mathcal{B - A}},\;t\in\mathbb{R} \\
            \text{Line}\parenthesisA{t} &= \parenthesisA{1 - t}\mathcal{A} + t\mathcal{B} \\
            \text{Line}\parenthesisA{t} &= \bracketA{\parenthesisA{1 - t}x_0 + ty_0, \dots, \parenthesisA{1 - t}x_n + ty_n}
        \end{align*}
        \vspace{-15pt}
        \item The \textbf{derivative} of a curve is the \textbf{tangent vector} at a given point.
        \item Given the \textbf{weights} $a_i$ and the \textbf{basis functions} $b_i$, for $i\in\bracketA{0, n}$, a \textbf{polynomial curve} can be represented as
        \begin{align*}
            q\parenthesisA{t} = a_0b_0\parenthesisA{t} + \dots + a_nb_n\parenthesisA{t} = \sum_{k = 0}^{n} a_kb_k\parenthesisA{t}
        \end{align*}
        For example, a line would be $b_0 = \parenthesisA{1 - t}$, $b_1 = t$, and $b_{i \geq 2} = 0.$
        \item \textbf{Relevant} curves are \textbf{degree 3} (also know as \textbf{cubics}) given that 3 is the lowest degree that can represented an \textbf{S-shape} (shape with an inflection point). Since the cubic curve
        \begin{align*}
            q\parenthesisA{t} = \bracketA{f_0\parenthesisA{t}, f_1\parenthesisA{t}, f_2\parenthesisA{t}},\;f_u\parenthesisA{t} = x_u + y_ut + z_ut^2 + w_ut^3
        \end{align*}
        suffers from too many coefficients to define a shape, a solution is to use \textbf{control points}.
        \begin{customcenter}[0pt]
            \begin{tabular}{|l|c|c|c|c|}
                \hline
                \diagbox[width=2.1cm]{Schema}{Basis} & $b_0$ & $b_1$ & $b_2$ & $b_3$ \\
                \hline
                \textbf{Bezier} & $\parenthesisA{1 - t}^3$ & $3t\parenthesisA{1 - t}^2$ & $3t^2\parenthesisA{1 - t}$ & $t^3$ \\
                \hline
                \textbf{Hermite} & $2t^3\hspace{-2pt}-\hspace{-1pt}3t^2\hspace{-2pt}+\hspace{-1pt}1$ & $-2t^3\hspace{-2pt}+\hspace{-1pt}3t^2$ & $t^3\hspace{-2pt}-\hspace{-1pt}2t^2\hspace{-2pt}+ t$ & $t^3\hspace{-2pt}-\hspace{-1pt}t^2$ \\            
                \hline
            \end{tabular}
        \end{customcenter}
    \end{itemize}
\end{minipage}

