%%%%%%%%%% Page 2 %%%%%%%%%%
\newpage
\colorfulheader{computer science}
\begin{customcenter}[0pt]
    \begin{customcenter}[2pt]
        \textbf{\textcolor{white}{\hspace{2em}}C Memory Layout Comparison}
    \end{customcenter}
    \begin{tabular}{|l|c|c|}
         \hline
         & \textbf{Stack} & \textbf{Heap} \\
         \hline
         \textbf{Memory} & Allocated in contiguous block & Allocated in any random order \\ 
         \hline
         \textbf{Allocatio$|$Deallocation} & Automatic by compiler & Manual by programmer \\
         \hline
         \textbf{Cost} & Less & More \\
         \hline
         \textbf{Implementation} & Easy & Hard \\
         \hline
         \textbf{Access Time} & Faster & Slower \\
         \hline
         \textbf{Main Issue} & Small memory & Memory fragmentation \\
         \hline
         \textbf{Safety} & Thread safe & Not thread safe (data visible to all threads) \\
         \hline
         \textbf{Data Type} & Linear & Hierarchical \\
         \hline
    \end{tabular}
    \emptyline
    \vspace{5pt}
    \begin{customcenter}[2pt]
    \textbf{\textcolor{white}{\hspace{2em}}Data Structures Complexity}
    \end{customcenter}
    \begin{tabular}{|l|c|c|c|c|c|}
         \hline
         & Accessing & Search & Insertion & Deletion & Space \\
         \hline
         Array & \DSC{1} & \DSC{n} & \DSC{n} & \DSC{n} & \DSC{n} \\
         \hline
         Stack (LIFO) & \DSC{n} & \DSC{n} & \DSC{1} & \DSC{1} & \DSC{n} \\
         \hline
         Queue (FIFO) & \DSC{n} & \DSC{n} & \DSC{1} & \DSC{1} & \DSC{n} \\
         \hline 
         Linked List & \DSC{n} & \DSC{n} & \DSC{1} & \DSC{1} & \DSC{n} \\
         \hline
         Hash Table & & \DSC{1} & \DSC{1} & \DSC{1} & \DSC{n} \\
         \hline
         Binary Search Tree$^*$ & \DSC{h} & \DSC{h} & \DSC{h} & \DSC{h} & \DSC{n} \\
         \hline
         Binary Heap & Min Heap \DSC{1} & \DSC{log\text{ }n} & \DSC{log\text{ }n} & \DSC{log\text{ }n} & \DSC{n} \\
         & Max Heap \DSC{1} & & & & \\
         \hline
    \end{tabular}
    \\
    $^*h$ would be the height of the tree. If tree is balanced $h = log\text{ } n$
\end{customcenter}
\vspace{5pt}
%%%%%%%%%% Page 2 - Col 1 %%%%%%%%%%
\begin{minipage}[t]{0.485\textwidth}
    \colorfulsection{Dynamic Programming}\\
    \emptyline
    Programming paradigm to \textbf{efficiently} explore all possible solutions to a problem. The problem should have one of the following characteristics
    \begin{itemize}
        \setlength\itemsep{0pt}
        \item \underline{Problem can be broken down in \textbf{overlapping subproblems}}
        
        A rule of thumb is to notice if \textbf{future decisions depend on earlier decision}        
        \item \underline{Problem has an \textbf{optimal substructure}}

        For example the problem would ask optimal value (min or max) of something, here are some sample phrases
        \begin{itemize}
            \item What is the minimum cost of...
            \item Compute the maximum profit from...
            \item How many ways can you...
            \item Find the longest possible...
        \end{itemize}
    \end{itemize}
    Care has to be taken to not confuse a dynamic programming (DP) problem with \textbf{greedy} problems. Both would have optimal substructure but greedy does not have overlapping characteristic.\\
    \emptyline
    Two ways of implementing a DP algorithm,\\ 
    \textbf{Top-Down} (recursion) or \textbf{Bottom-Up} (iterative).\\
    
    An important optimization in the top-down approach is \textbf{memoization} that stored previously computed subresults (commonly in a hash) to be re-used in the future, avoiding recomputation.
    \begin{customcenter}[8pt]
        \begin{tabular}{|c|c|c|}
            \hline
            & \textbf{Top-Down} & \textbf{Bottom-Up} \\
            \hline
            Pro & Easy to write & Runs faster\\
            \hline
            Con & Overhead & Order of operations matter\\
            \hline
        \end{tabular}
    \end{customcenter}
    In DP one can define a \textbf{state} as a \textbf{set of variables that can sufficiently describe a scenario}.
\end{minipage}
%%%%%%%%%%%%%%%%%%%%%%%%%%%%%%%%%%%%
\hspace{5pt}
%%%%%%%%%% Page 2 - Col 2 %%%%%%%%%%
\begin{minipage}[t]{0.485\textwidth}
    To \quotes{\textit{produce}} an algorithm for a DP problem, one can follow the next steps:
    \begin{enumerate}
        \setlength\itemsep{0pt}
        \item Create a \textbf{function} or \textbf{data structure} that will compute the answer of the problem for \textbf{every given state}
        \item A \textbf{recurrence relation} to transition between states
        \item Define \textbf{bases cases} to avoid inifinite recursion or faulty iterations
    \end{enumerate}
    For example, the following \textbf{climbing stairs} problem
    \begin{tcolorbox}
        You are climbing a staircase. It takes $n$ steps to reach the top. Each time you can either climb $1$ or $2$ steps. \textbf{How many distinct ways} can you climb to the top?
    \end{tcolorbox}

    Thus, the DP steps for above problem would be
    \begin{enumerate}
        \setlength\itemsep{0pt}
        \item \textbf{Function} $dp(i)$, where $i$ represents how many ways to climb to the $i^{\text{th}}$ step.
        \item \textbf{Recurrence} $dp(i) = dp(i - 1) + dp(i - 2)$, since according to the description one can climb either $1$ or $2$ steps each time.
        \item \textbf{Base Case} $dp(1) = 1$ and $dp(2) = 2$, since there is 1 way (1 step) to climb to step 1 and 2 ways (1 step + 1 step and 2 step) to climb to step 2.
    \end{enumerate}
    A minimal implementation in C\texttt{++} of above steps is
    \begin{lstlisting}
        int dpRecursive(int i)
        {
            if (i <= 2) return i; // base case
            // recurrence
            return dpRecursive(i - 1) + dpRecursive(i - 2);
        }
        int countSteps(int n) { return dpRecursive(n); }
    \end{lstlisting}
    Although the code would solve the problem, it has poor complexities, $\mathcal{O}\parenthesisA{2^n}$ time and $\mathcal{O}\parenthesisA{n}$ space.
\end{minipage}
